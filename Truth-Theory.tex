\section{Truth Theoretic Interpretation}

For this part, we work in the category $Rel$ of sets and relations.  We take $U$ to be a universal reference set of individuals  and take $N$ to be the set of all  subsets of $U$, denoted by ${\cal P}(U)$. A common noun is modelled by the set of all subsets of  its individuals.    We take $S$ to be the singleton set $I = \{\star\}$, that is the unit of tensor product in $Rel$.  A verb is the set of all  subsets  of a relation (corresponding to its predicate). For an intransitive verb, this relation is on the set $N \times S$; since we have $N \times S \cong N$,  each relation corresponds to a subset of $N$.  For a transitive verb, it is a relation on the set $N \times S \times N \cong N \times N$. 



The map $Det$ sends a subset of individuals to a set of its subsets exactly in the same way as defined by \cite{BarwiseCooper81}. For example, for $Det$ = `two', the output is the set of subsets  of individuals whose elements have cardinality exactly two; for $Det$ = `some', the output is the set of  subsets of individuals and so on. 

The truth-theoretic meaning of the sentence ``Det Sbj Verb''  is obtained  by computing a simplified version of the   morphism  developed in section 4 in category $Rel$. The simplification is because we have $S = I$ and hence the morphisms that are applied to object $S$ can be dropped. 

Suppose we have:
\begin{eqnarray*}
\overline{\semantics{\text{Sbj}}} &=& \{A_i \mid A_i \in {\cal P}(\semantics{\text{Sbj}})\}\\
\overline{\semantics{\text{Verb}}} &=& \{B_j \mid B_j \in {\cal P}(\semantics{\text{Verb}})\}\\
Det\Big(\overline{\semantics{\text{Sbj}}}\Big) &=& \{D_k \mid D_k \in Det(\semantics{\text{Sbj}})\}
\end{eqnarray*}
where   $\semantics{\text{Sbj}} \subseteq U$ and $\semantics{\text{Verb}} \subseteq U$ are the set-theoretic meanings of ``Sbj'' and ``Verb'', and $Det(S)$  is the same as in the generalised quantifier approach. 

The compact closed meaning of a quantified sentence is computed in three steps as follows. In the  first step, we obtain:

\begin{align*}
(\delta_N \otimes 1_{N})\Big(\overline{\semantics{\text{Sbj}}} \otimes \overline{\semantics{\text{Verb}}}\Big) =  \{(A_i, A_i) \mid A_i \in {\cal P} (\semantics{\text{Sbj}})\} \otimes \{B_j \mid B_j \in  {\cal P} (\semantics{\text{Verb}})\}
\end{align*}


\noindent
In the second step, we obtain:

\begin{align*}
(Det \otimes  \mu_N)\Big(\{(A_i, A_i) \mid A_i \in {\cal P} (\semantics{\text{Sbj}})\} \otimes \{B_j \mid B_j \in  {\cal P} (\semantics{\text{Verb}})\}\Big) &=\\
Det\Big(\{A_i \mid A_i \in {\cal P} (\semantics{\text{Sbj}})\}\Big) \otimes \{A_i \mid A_i = B_j, A_i \in {\cal P} (\semantics{\text{Sbj}}), B_j \in  {\cal P}(\semantics{\text{Verb}})\} &=\\
\{D_k \mid D_k \in Det(\semantics{\text{Sbj}})\} \otimes \{A_i \mid A_i = B_j, A_i \in {\cal P} (\semantics{\text{Sbj}}), B_j \in  {\cal P}(\semantics{\text{Verb}})\}
\end{align*}

\noindent
In the final step, we obtain:
\begin{align*}
\epsilon\Big(\{D_k \mid D_k \in Det(\semantics{\text{Sbj}})\} \otimes \{A_i \mid A_i = B_j, A_i \in {\cal P} (\semantics{\text{Sbj}}), B_j \in  {\cal P}(\semantics{\text{Verb}})\}\Big) &=\\
 \{\star \mid  D_k = A_i, D_k \in Det(\semantics{\text{Sbj}}), A_i = B_j, A_i \in {\cal P} (\semantics{\text{Sbj}}), B_j \in  {\cal P} (\semantics{\text{Verb}})\}
\end{align*}




Recalling that, as shown in \cite{CoeckePaquettePavlovic09,CoeckePaq},  the Frobenius $ \mu$ map is the analog of  set-theoretic intersection and the compact closed  epsilon map is the analog of  set-theoretic application, it is not hard to show that  the truth-theoretic interpretation of the compact closed semantics of quantified sentences provides us with the same meaning as their generalised quantifier semantics. In this section we make this formal as follows.

\begin{definition}
\label{deftrue}
The compact closed meaning of the sentence ``Q Sbj Verb'' is true  in $Rel$ if and only if either $Det(\semantics{\text Sbj}) \neq \{\emptyset\}$ and we have 
\[
\epsilon \circ (Det \otimes \mu) \circ (\sigma \otimes 1_N) \Big(\overline{\semantics{\text{Sbj}}} \otimes \overline{\semantics{\text{Verb}}} \Big) = \{\star\} 
\] 
or  $Det(\semantics{\text Sbj}) = \{\emptyset\}$,  and we have the above as well as the following: 
\[
\mu \Big(\overline{\semantics{\text{Sbj}}} \otimes \overline{\semantics{\text{Verb}}} \Big) = \{\emptyset\}
\]
\end{definition}


\begin{proposition}
The  compact closed meaning of a quantified sentence computed in $Rel$ is true if and only if  its generalised quantifier meaning is true.
\end{proposition}

\begin{proof}
For the right to left direction,  suppose  the generalised quantifier meaning of the sentence is true, that is $\semantics{\text{Sbj}} \cap \semantics{\text{Verb}} \in Det(\semantics{\text{Sbj}}$ and consider the case when  $Det(\semantics{\text Sbj}) \neq \{\emptyset\}$. We have to show that $\epsilon \circ (Det \otimes \mu) \circ (\sigma \otimes 1_N) \Big(\overline{\semantics{\text{Sbj}}} \otimes \overline{\semantics{\text{Verb}}} \Big) = \{\star\}$.  For this, we need to show that there is a set $G$ equal to $D_k= A_i = B_j$ such that $G$ is in $Det(\semantics{\text{Sbj}})$ and $ {\cal P} (\semantics{\text{Sbj}})$ and  ${\cal P} (\semantics{\text{Verb}})$.  Take $G$ to be $\semantics{\text{Sbj}} \cap \semantics{\text{Verb}}$.  Then, it is a subset of   $\semantics{\text{Sbj}}$, hence an element of  $ {\cal P} (\semantics{\text{Sbj}})$, a subset of  $ \semantics{\text{Verb}}$, hence an element of $ {\cal P} (\semantics{\text{Verb}})$, and an element of $Det(\semantics{\text{Sbj}}$. 

Now consider  the case where   $Det(\semantics{\text Sbj}) = \{\emptyset\}$, the above argument still holds and we have that $G = \emptyset$.  It remains to check whether $\mu\Big(\overline{\semantics{\text{Sbj}}} \otimes \overline{\semantics{\text{Verb}}} \Big) = \{\emptyset\}$. This is indeed the case,   since $D_k = A_i = B_j = \emptyset$, hence   $\mu\Big(\overline{\semantics{\text{Sbj}}} \otimes \overline{\semantics{\text{Verb}}} \Big) = \{A_i \mid A_i = B_j, A_i \in {\cal P} (\semantics{\text{Sbj}}), B_j \in  {\cal P}(\semantics{\text{Verb}})\} = 
\{\emptyset\}$. 

For the left to right direction, suppose the compact closed meaning of the sentence is true in $Rel$, and consider  the case when $Det(\semantics{\text Sbj}) \neq \{\emptyset\}$. Then  we have  subsets $D_k = A_i = B_j$ such that $D_k \in Det(\semantics{\text Sbj}), A_i \in {\cal P}(\semantics{\text Sbj})$, and $B_j \in {\cal P}(\semantics{\text Verb})$.  Pick an arbitrary such subset, e.g.  $G$, such that it   equal to  $D_k = A_i = B_j$; we have that $G \in   {\cal P}(\semantics{\text Sbj})$, hence $G \subseteq \semantics{\text Sbj}$; and that $G \in  {\cal P}(\semantics{\text Verb})$, hence $G \subseteq \semantics{\text Verb}$. It thus follows that $G \subseteq \semantics{\text Sbj}  \cap \semantics{\text Verb}$. At the same time $G \in Det(\semantics{\text Sbj})$, hence the generalised quantifier meaning is also true.  

For the case when $Det(\semantics{\text Sbj}) = \{\emptyset\}$, we still have the above, so the generalised quantifier meaning of the sentence remains true. In this case we moreover  have that $\mu\Big(\overline{\semantics{\text{Sbj}}} \otimes \overline{\semantics{\text{Verb}}} \Big) = \{\emptyset\}$, which implies that  $G= A_i = B_j = D_k = \emptyset$. If the second condition did not hold, we would have that $\mu\Big(\overline{\semantics{\text{Sbj}}} \otimes \overline{\semantics{\text{Verb}}} \Big) = \{A_i \mid A_i = B_j, A_i \in {\cal P} (\semantics{\text{Sbj}}), B_j \in  {\cal P}(\semantics{\text{Verb}})\}$ is equal to a set of subsets, at least one of which is non empty, e.g.  $\{X_1, X_2, \cdots\}$, where for  $X_w$ it holds that $X_w \neq \emptyset$, and $X_w \subseteq \semantics{\text Sbj}$ and $X_w \subseteq \semantics{\text Verb}$. It would then follows that  $X_w = \semantics{\text Sbj} \cap \semantics{\text Verb} \neq \emptyset$;  whereas $\Det(\semantics{\text Sbj}) = \{\emptyset\}$, hence the generalised quantifier meaning of the sentence would become false. 


\end{proof}

\bigskip
\noindent
{\bf Example.} As a truth-theoretic  example,  suppose we have two male individuals $m_1, m_2$  and a cat   individual $c_1$.  Suppose further that  the verb `sneeze'  applies to individuals $m_1$ and $c_1$. Hence, we have the following interpretations for the lemmas of words ``man'', ``cat'', and ``sneeze'':

\[
\overline{\semantics{\text{men}}} =  \{\emptyset,  \{m_1\}, \{m_2\}, \{m_1, m_2\}\}  \qquad
\overline{\semantics{\text{cat}}} =  \{\emptyset, \{c_1\}\}  \qquad
\overline{\semantics{\text{sneeze}}} = \{\emptyset, \{m_1\}, \{c_1\}, \{m_1, c_1\}\}
\]

\noindent
Consider the  following quantified phrases and their interpretations:

\[
Some\Big(\overline{\semantics{\text{men}}}\Big) =  \{\{m_1\}, \{m_2\}, \{m_1, m_2\}\} \qquad
One\Big(\overline{\semantics{\text{man}}}\Big) = \{\{m_1\}, \{m_2\}\} \qquad 
No\Big(\overline{\semantics{\text{men}}}\Big) = \{\emptyset\}
\]

\noindent
In the first step of computation of the meaning of  ``some men sneeze'', we obtain:

\begin{align*}
(\delta_N \otimes 1_{N})\Big(\overline{\semantics{\text{men}}} \otimes \overline{\semantics{\text{sneeze}}}\Big) =&\\
  \{(\{\emptyset\}, \{\emptyset\}), (\{m_1\}, \{m_1\}), (\{m_2\}, \{m_2\}), (\{m_1, m_2\}, \{m_1, m_2\}) \} \otimes  \{\emptyset, \{m_1\}, \{c_1\}, \{m_1, c_1\}\} 
\end{align*}

\noindent
In the second step, we obtain:
\begin{align*}
\Big(Some \otimes \mu\Big) \Big ( \{(\{\emptyset\}, \{\emptyset\}), (\{m_1\}, \{m_1\}), (\{m_2\}, \{m_2\}), (\{m_1, m_2\}, \{m_1, m_2\}) \} \otimes  \{\emptyset, \{m_1\}, \{c_1\}, \{m_1, c_1\}\}  \Big) =&\\
Some\Big ( \{\{m_1\}, \{m_2\}, \{m_1, m_2\}\} \Big) \otimes \mu \Big( \{\emptyset, \{m_1\}, \{m_2\}, \{m_1, m_2\}\}  \otimes \{\emptyset, \{m_1\}, \{c_1\}, \{m_1, c_1\}\} \Big)=&\\
\{\{m_1\}, \{m_2\}, \{m_1, m_2\}\}  \otimes \{\emptyset, \{m_1\}\}
\end{align*}

\noindent
In the last step, we obtain the following via the  relation $\epsilon \colon N \times N \to \{\star\}$ being $\{((\{m_1\}, \{m_1\}), \star)\}$:

\begin{align*}
\epsilon\Big(\{\{m_1\}, \{m_2\}, \{m_1, m_2\}\}  \otimes \{\emptyset, \{m_1\}\}\Big) = \{\star\}
\end{align*}

\noindent
Hence, the meaning of the sentence is true.  For the sentence ``One man sneezes'',  the second and third steps of computation are as follows:

\begin{align*}
\epsilon\Big( One\Big ( \{\{m_1\}, \{m_2\}, \{m_1, m_2\}\} \Big) \otimes \mu \Big( \{\emptyset, \{m_1\}, \{m_2\}, \{m_1, m_2\}\}  \otimes \{\emptyset, \{m_1\}, \{c_1\}, \{m_1, c_1\}\} \Big)\Big) =&\\
\epsilon \Big(\{\{m_1\}, \{m_2\}\}  \otimes \{\emptyset, \{m_1\}\}\Big) = \{\star\}
\end{align*}

\noindent
So the meaning of this sentence is also true (it has the same $\epsilon$ relation as the previous case). Now consider the case of the  sentence ``no man sneezes'' in which case $No (\semantics{\text man}) = \emptyset$. In this case we obtain  the following at the final step of computation

\begin{align*}
\epsilon\Big( No\Big ( \{\{m_1\}, \{m_2\}, \{m_1, m_2\}\} \Big) \otimes \mu \Big( \{\emptyset, \{m_1\}, \{m_2\}, \{m_1, m_2\}\}  \otimes \{\emptyset, \{m_1\}, \{c_1\}, \{m_1, c_1\}\} \Big)\Big) =&\\
\epsilon \Big(\{\emptyset\}  \otimes \{\emptyset, \{m_1\}\}\Big) = \{\star\}
\end{align*}

\noindent
However, this is not a true sentence, since  
\[
\mu(\overline{\semantics{\text men}} \otimes \overline{\semantics{\text sneeze}}) = \mu(\{\emptyset, \{m_1\}\} \otimes \{\emptyset, \{m_1\}, \{c_1\}, \{m_1, c_1\}\}) = \{\emptyset, \{m_1\}\}   \neq \{\emptyset\}
\]
 If we had $\overline{\semantics{\text dog}} = \{\emptyset, \{d_1\}\}$, then the compact closed meaning of the sentence ``No dogs sneeze''  would become true, since we would have
\[
\mu(\overline{\semantics{\text dogs}} \otimes \overline{\semantics{\text sneeze}}) = \mu(\{\emptyset, \{d_1\}\} \otimes \{\emptyset, \{m_1\}, \{c_1\}, \{m_1, c_1\}\}) = \{\emptyset\}
\]
and also that  
\[
\epsilon(No \otimes \mu)(\sigma \otimes 1) (\overline{\semantics{\text dogs}} \otimes \overline{\semantics{\text sneeze}})=  \epsilon(\{\emptyset\}  \otimes \{\emptyset\}) = \{\star\}
\]
Hence ``no dogs sneeze'' is true. 

\section{Truth-Theory in Vector Spaces} 

One can do the same calculations as in $Rel$ in $FVect$ and obtain the same truth theoretic meanings in vector spaces. In this case, all we have to do is to model a set $N = \{n_1, n_2, \cdots\}$, by a vector space $V_N$ spanned by $N$, that is $V_N = \{\ov{n}_i\}_i$. In the case of our model, $N = {\cal P}(U)$, for $U$ a set of individuals, hence $V_N$ is spanned by subsets of $U$.  Denoting these subsets by $U_i$, we have $V_{{\cal P}(U)} = \{\ov{U}_i\}_i$. The one element set $\{\star\}$ is then modelled  by the one dimensional vector space $\{\ov{1}\}$, which  models the sentence space, that is we have $S = \{\ov{1}\} = V_{\{\star\}}$.  The zero element set, that is the empty set, is modelled by the zero vector $\ov{0}$. 

We demonstrate the computation for the truth theoretic meaning of ``Q Sbj Verb'' in the above vector space below. For the meanings of the words therein, we have:

\begin{eqnarray*}
\ov{\text{Sbj}} &=& \sum_i \ov{U}_i \qquad \text{for} \quad U_i \in {\cal P}(\semantics{\text{Sbj}})\\
\ov{\text{Verb}} &=& \sum_j \ov{U}_j  \otimes \{\star\} \cong \sum_i \ov{U}_j  \qquad \text{for} \quad  U_j \in {\cal P}(\semantics{\text{Verb}})\\
Det\Big(\ov{\text{Sbj}}\Big) &=& \sum_k \ov{U}_k \qquad \text{for} \quad   U_k \in Det(\semantics{\text{Sbj}})
\end{eqnarray*}




The first step of the computation  is as follows:


\begin{align*}
(\delta_N \otimes 1_{N})\Big(\ov{\text Sbj} \otimes \ov{\text Verb}\Big) =   (\delta_N \otimes 1_{N})\Big(\sum_i \ov{U}_i  \otimes \sum_j \ov{U}_j \Big ) = (\sum_i \ov{U}_i \otimes \ov{U}_i) \otimes (\sum_j \ov{U}_j)
\end{align*}

\noindent
In the second step, we obtain:

\begin{align*}
(Det \otimes  \mu_N)\Big(\sum_i \ov{U}_i \otimes \ov{U}_i) \otimes (\sum_j \ov{U}_j\Big) =  Det(\sum_i \ov{U}_i) \otimes  (\sum_i \sigma_{ij} \ov{U}_i)   = \sum_k \ov{U}_k \otimes \sum_i \sigma_{ij} \ov{U}_i
\end{align*}

\noindent
The final step  provides us with the following:

\begin{align*}
(\epsilon_{N})  \Big(\sum_k \ov{U}_k \otimes \sum_i \sigma_{ij} \ov{U}_i  \Big) =  \sum_{ijk}  \langle  \ov{U}_k   \mid  \sigma_{ij} \ov{U}_i \rangle  
\end{align*}

\begin{definition}
\label{deftrue}
The vector space meaning of the sentence ``Q Sbj Verb'' in $FVect$  is true if and only if either $Det(\ov{\text Sbj}) \neq \ov{0}$ and we have: 
\[
\epsilon \circ (Det \otimes \mu) \circ (\sigma \otimes 1_N) \Big(\ov{\text{Sbj}} \otimes \ov{\text{Verb}} \Big) \geqslant \ov{1} 
\] 
or  $Det(\ov{\text Sbj}) = \ov{0}$ and we have the above as well as the following:
\[
\mu \Big(\ov{\text{Sbj}} \otimes \ov{\text{Verb}} \Big) = \ov{0}\]
\end{definition}

\begin{proposition}
The  compact closed meaning of a quantified sentence computed in $FVect$  is true if and only if  its generalised quantifier meaning is true.
\end{proposition}

\begin{proof}
For the right to left direction, suppose the generalised quantifier meaning is true, then we have a set $G \subseteq U$, such that  $G\subseteq\semantics{\text Sbj}, G \subseteq \semantics{\text Verb}$, and $G \in Det (\semantics{\text Sbj})$. Hence, $G$ is modelled by the vector $\ov{G} \in V_{{\cal P}(U)}$, which moreover has the property that $\ov{G} \in \ov{\text Sbj}, \ov{G} \in \ov{\text Verb}$, and $\ov{G} \in Det(\ov{\text Sbj})$. This implies that $\langle \ov{G} \mid \ov{G}\rangle = \ov{1}$, hence the compact closed meaning of the sentence in $FVect$ is  true. 
\end{proof}

\bigskip
\noindent
{\bf Example.}
As an example,  consider  the  meaning  of  ``some men sneeze'', in the first step of the computation we have:
\begin{eqnarray*}
&&(\delta \otimes 1_N) \Big(\ov{\text men} \otimes \ov{\text sneeze}\Big) = (\delta \otimes 1) \Big((\emptyset + \{m_1\} + \{m_2\} + \{m_1, m_2\}) \otimes (\emptyset + \{m_1\} + \{c_1\} + \{m_1, c_1\}) \Big)\\
&&=  (\emptyset \otimes \emptyset + \{m_1\} \otimes \{m_1\} + \{m_2\} \otimes \{m_2\} + \{m_1, m_2\} \otimes \{m_1, m_2\}) \otimes (\emptyset + \{m_1\} + \{c_1\} + \{m_1, c_1\})
\end{eqnarray*}

\noindent
In the second step, we apply $(Some \otimes \mu)$ to the above and obtain:
\begin{eqnarray*}
&& Some\Big(\emptyset + \{m_1\} + \{m_2\} + \{m_1, m_2\}\Big) \otimes \mu \Big((\emptyset  + \{m_1\}  + \{m_2\}  + \{m_1, m_2\}) \otimes (\emptyset + \{m_1\} + \{c_1\} + \{m_1, c_1\})\Big) \\
&&= (\{m_1\} + \{m_2\} + \{m_1, m_2\}) \otimes (\emptyset + \{m_1\})
\end{eqnarray*}

\noindent
In the final step, we apply $\epsilon$ to the above and obtain:
\[
\langle \{m_1\} + \{m_2\} + \{m_1, m_2\} \mid \emptyset + \{m_1\} \rangle =  1
\]

\noindent
For ``one man sneezes'' the hole computation is as follows:
\begin{eqnarray*}
&&\epsilon(One \otimes \mu)(\delta \otimes 1_N) \Big(\ov{\text men} \otimes \ov{\text sneeze}\Big) = \\
&&  \epsilon(One \otimes \mu)(\emptyset \otimes \emptyset + \{m_1\} \otimes \{m_1\} + \{m_2\} \otimes \{m_2\} + \{m_1, m_2\} \otimes \{m_1, m_2\}) \otimes (\emptyset + \{m_1\} + \{c_1\} + \{m_1, c_1\})\\
&&= \epsilon \Big (One\big(\emptyset + \{m_1\} + \{m_2\} + \{m_1, m_2\}\Big) \otimes \mu \Big((\emptyset  + \{m_1\}  + \{m_2\}  + \{m_1, m_2\}) \otimes (\emptyset + \{m_1\} + \{c_1\} + \{m_1, c_1\})\Big) \\
&&= \epsilon \Big((\{m_1\} + \{m_2\} ) \otimes (\emptyset + \{m_1\})\Big)\\
&&= \langle \{m_1\} + \{m_2\}  \mid \emptyset + \{m_1\} \rangle =  1
\end{eqnarray*}


\noindent
For ``no man sneezes'', we have that $Det(\ov{\text Sbj}) = 0$, hence for this to be true its $FVect$ meaning should be above 1 and also $\mu \Big(\ov{\text{Sbj}} \otimes \ov{\text{Verb}} \Big)$ should be the $\emptyset$ vector. 

The first step of the computation  of the $FVect$ meaning of the sentence is as before, in the second and third steps we obtain:

\begin{eqnarray*}
&&\epsilon \Big (No (\emptyset + \{m_1\} + \{m_2\} + \{m_1, m_2\}) \otimes  \mu\big(\emptyset  + \{m_1\}  + \{m_2\}  + \{m_1, m_2\}) \otimes (\emptyset + \{m_1\} + \{c_1\} + \{m_1, c_1\}\big)\Big)\\
&&= 
\epsilon \Big(\{\emptyset\} \otimes (\emptyset + \{m_1\}) \Big) = 1
\end{eqnarray*}

\noindent
But this sentence is not true, since we have 
\[
(1_N \otimes \mu) \circ (\sigma \otimes 1_N)(\ov{\text men} \otimes \ov{\text sneeze})   = \emptyset + \{m_1\} \neq \emptyset
\]
Whereas, for  ``no dogs sneeze'', where $\ov{\text dogs} = \emptyset + \{d_1\}$, we will have:
\[
\mu\big(\ov{\text dogs} \otimes \ov{\text sneeze}\big) = \mu \big((\emptyset + \{d_1\}) \otimes (\emptyset + \{m_1\} + \{c_1\} + \{m_1, c_1\})\big) = \emptyset
\]
And also that 
\[\epsilon(No \otimes \mu) (\delta \otimes 1)(\ov{\text dogs} \otimes \ov{\text sneeze}) =
\epsilon(\emptyset \otimes \emptyset) = 1
\]

\noindent
Hence ``no dogs sneeze'' is true. 

%%% Local Variables: 
%%% mode: latex
%%% TeX-master: "Quant"
%%% End: 
