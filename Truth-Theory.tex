\section{Truth Theoretic Interpretation}

For this part, we work in the category $Rel$ of sets and relations.  We take $U$ to be a universal reference set of individuals  and take $N$ to be the set of all  subsets of $U$, denoted by ${\cal P}(U)$. A common noun is modelled by the set of all subsets of  its individuals.    We take $S$ to be the singleton set $I = \{\star\}$, that is the unit of tensor product in $Rel$.  A verb is the set of all  subsets  of a relation (corresponding to its predicate). For an intransitive verb, this relation is on the set $N \times S$; since we have $N \times S \cong N$,  each relation corresponds to a subset of $N$.  For a transitive verb, it is a relation on the set $N \times S \times N \cong N \times N$. 



The map $Det$ sends a subset of individuals to a set of its subsets exactly in the same way as defined by \cite{BarwiseCooper81}. For example, for $Det$ = `two', the output is the set of subsets  of individuals whose elements have cardinality exactly two; for $Det$ = `some', the output is the set of  subsets of individuals and so on. 

The truth-theoretic meaning of the sentence ``Det Sbj Verb''  is obtained  by computing a simplified version of the   morphism  developed in section 4 in category $Rel$. The simplification is because we have $S = I$ and hence the morphisms that are applied to object $S$ can be dropped. 


\bigskip
\noindent
{\bf Formal Justification.} 
Recalling that, as shown in \cite{CoeckePaquettePavlovic09,CoeckePaq},  the Frobenius $ \mu$ map is the analog of  set-theoretic intersection and the compact closed  epsilon map is the analog of  set-theoretic application, it is not hard to show that  the truth-theoretic interpretation of the compact closed semantics of quantified sentences provides us with the same meaning as their generalised quantifier semantics. In this section we make this formal as follows.

\begin{definition}
\label{deftrue}
The compact closed meaning of the sentence ``Q Sbj Verb'' is true if and only if $Det(\semantics{\text Sbj}) \neq \{\emptyset\}$ and we have 
\[
\epsilon \circ (Det \otimes \mu) \circ (\sigma \otimes 1_N) \Big(\overline{\semantics{\text{Sbj}}} \otimes \overline{\semantics{\text{Verb}}} \Big) = \{\star\} 
\] 
For the case when  $Det(\semantics{\text Sbj}) = \{\emptyset\}$,  we have to have the above as well as that the second element of the pair in  the result of  $(1_N \otimes \mu) \circ (\sigma \otimes 1_N) \Big(\overline{\semantics{\text{Sbj}}} \otimes \overline{\semantics{\text{Verb}}} \Big)$ is  the singleton set $\{\emptyset\}$. 

In either of the above cases, we have:
\begin{eqnarray*}
\overline{\semantics{\text{Sbj}}} &=& \{A_i \mid A_i \in {\cal P}(\semantics{\text{Sbj}})\}\\
\overline{\semantics{\text{Verb}}} &=& \{B_j \mid B_j \in {\cal P}(\semantics{\text{Verb}})\}\\
Det\Big(\overline{\semantics{\text{Sbj}}}\Big) &=& \{D_k \mid D_k \in Det(\semantics{\text{Sbj}})\}
\end{eqnarray*}
where   $\semantics{\text{Sbj}} \subseteq U$ and $\semantics{\text{Verb}} \subseteq U$ are the set-theoretic meanings of ``Sbj'' and ``Verb'', and $Det(S)$  is the same as in the generalised quantifier approach. 
\end{definition}

The compact closed meaning of a quantified sentence is computed in three steps as follows. In the  first step, we obtain:

\begin{align*}
(\delta_N \otimes 1_{N})\Big(\overline{\semantics{\text{Sbj}}} \otimes \overline{\semantics{\text{Verb}}}\Big) =  \{(A_i, A_i) \mid A_i \in {\cal P} (\semantics{\text{Sbj}})\} \otimes \{B_j \mid B_j \in  {\cal P} (\semantics{\text{Verb}})\}
\end{align*}


\noindent
In the second step, we obtain:

\begin{align*}
(Det \otimes  \mu_N)\Big(\{(A_i, A_i) \mid A_i \in {\cal P} (\semantics{\text{Sbj}})\} \otimes \{B_j \mid B_j \in  {\cal P} (\semantics{\text{Verb}})\}\Big) &=\\
Det\Big(\{A_i \mid A_i \in {\cal P} (\semantics{\text{Sbj}})\}\Big) \otimes \{A_i \mid A_i = B_j, A_i \in {\cal P} (\semantics{\text{Sbj}}), B_j \in  {\cal P}(\semantics{\text{Verb}})\} &=\\
\{D_k \mid D_k \in Det(\semantics{\text{Sbj}})\} \otimes \{A_i \mid A_i = B_j, A_i \in {\cal P} (\semantics{\text{Sbj}}), B_j \in  {\cal P}(\semantics{\text{Verb}})\}
\end{align*}

\noindent
In the final step, we obtain:
\begin{align*}
\epsilon\Big(\{D_k \mid D_k \in Det(\semantics{\text{Sbj}})\} \otimes \{A_i \mid A_i = B_j, A_i \in {\cal P} (\semantics{\text{Sbj}}), B_j \in  {\cal P}(\semantics{\text{Verb}})\}\Big) &=\\
 \{\star \mid  D_k = A_i, D_k \in Det(\semantics{\text{Sbj}}), A_i = B_j, A_i \in {\cal P} (\semantics{\text{Sbj}}), B_j \in  {\cal P} (\semantics{\text{Verb}})\}
\end{align*}


\begin{proposition}
The  compact closed meaning of a quantified sentence  is true if and only if  its generalised quantifier meaning is true.
\end{proposition}

\begin{proof}
In the right to left direction,  first, consider the case when  $Det(\semantics{\text Sbj}) \neq \{\emptyset\}$.   Suppose $\semantics{\text{Sbj}} \cap \semantics{\text{Verb}} \in Det(\semantics{\text{Sbj}}$; we have to show that $\epsilon \circ (Det \otimes \mu) \circ (\sigma \otimes 1_N) \Big(\overline{\semantics{\text{Sbj}}} \otimes \overline{\semantics{\text{Verb}}} \Big) = \{\star\}$.  For this, we need to show that there is a set $D_k$ such that $D_k$ is in $Det(\semantics{\text{Sbj}})$ and $ {\cal P} (\semantics{\text{Sbj}})$ and  ${\cal P} (\semantics{\text{Verb}})$.  Suppose the latter is true and take a set $G$ such that $G$ is a subset of $\semantics{\text{Sbj}} \cap \semantics{\text{Verb}}$ and $G$ is  in $Det(\semantics{\text{Sbj}}$.  That is, $G$ is a subset of  both $\semantics{\text{Sbj}}$ and $ \semantics{\text{Verb}}$ and an element of $Det(\semantics{\text{Sbj}}$. This means that $G$ is an element of $ {\cal P} (\semantics{\text{Sbj}})$ and an element of  ${\cal P} (\semantics{\text{Verb}})$ and an element of $Det(\semantics{\text{Sbj}}$. And this means that  there is a set $D_k$ that makes the former set non-empty, namely $D_k = G$.  

For the case when   $Det(\semantics{\text Sbj}) = \{\emptyset\}$, the above argument still holds and we have that $D_k = G = \emptyset$.  In this case, it is indeed the case that the second element of  of the pair in $(1_N \otimes \mu) \circ (\sigma \otimes 1_N) \Big(\overline{\semantics{\text{Sbj}}} \otimes \overline{\semantics{\text{Verb}}} \Big) $ is $\{\emptyset\}$,  since $D_k = A_i = B_j = \emptyset$, hence   $\{A_i \mid A_i = B_j, A_i \in {\cal P} (\semantics{\text{Sbj}}), B_j \in  {\cal P}(\semantics{\text{Verb}})\} = 
\{\emptyset\}$. 

In the left to right direction, for the case when $Det(\semantics{\text Sbj}) \neq \{\emptyset\}$, the compact closed meaning of the sentence is the set $\{\star\}$ when we have  subsets $D_k = A_i = B_j$ such that $D_k \in Det(\semantics{\text Sbj}), A_i \in {\cal P}(\semantics{\text Sbj})$, and $B_j \in {\cal P}(\semantics{\text Verb})$.  Take an arbitrary such subset, e.g.  $G = D_k = A_i = B_j$, we have that $G \subseteq \semantics{\text Sbj}$ and $G \subseteq \semantics{\text Verb}$, hence $G \subseteq \semantics{\text Sbj}  \cap \semantics{\text Verb}$. At the same time $G \in Det(\semantics{\text Sbj})$, , hence the generalised quantifier meaning is also true.  

For the case when $Det(\semantics{\text Sbj}) = \{\emptyset\}$ and moreover $\{A_i \mid A_i = B_j, A_i \in {\cal P} (\semantics{\text{Sbj}}), B_j \in  {\cal P}(\semantics{\text{Verb}})\} = \{\emptyset\}$, we have that $A_i = B_j = D_k = \emptyset$, hence $G = \emptyset$ and again the generalised quantifier meaning is true. If the second condition did not hold, that is we had that $\{A_i \mid A_i = B_j, A_i \in {\cal P} (\semantics{\text{Sbj}}), B_j \in  {\cal P}(\semantics{\text{Verb}})\} = \{X_1, X_2, \cdots\}$, then we would have a set $X_w$ such that $X_w \subseteq \semantics{\text Sbj}$ and $X_w \subseteq \semantics{\text Verb}$,  hence $\semantics{\text Sbj} \cap \semantics{\text Verb} \neq \emptyset$;  whereas $\Det(\semantics{\text Sbj}) = \{\emptyset\}$, hence the generalised quantifier meaning of the sentence would become false. 


\end{proof}

\bigskip
\noindent
{\bf Example.} As a truth-theoretic  example,  suppose we have two male individuals $m_1, m_2$  and a cat   individual $c_1$.  Suppose further that  the verb `sneeze'  applies to individuals $m_1$ and $c_1$. Hence, we have the following interpretations for the lemmas of words ``man'', ``cat'', and ``sneeze'':

\[
\overline{\semantics{\text{men}}} =  \{\emptyset,  \{m_1\}, \{m_2\}, \{m_1, m_2\}\}  \qquad
\overline{\semantics{\text{cat}}} =  \{\emptyset, \{c_1\}\}  \qquad
\overline{\semantics{\text{sneeze}}} = \{\emptyset, \{m_1\}, \{c_1\}, \{m_1, c_1\}\}
\]

\noindent
Consider the  following quantified phrases and their interpretations:

\[
Some\Big(\overline{\semantics{\text{men}}}\Big) =  \{\{m_1\}, \{m_2\}, \{m_1, m_2\}\} \qquad
One\Big(\overline{\semantics{\text{man}}}\Big) = \{\{m_1\}, \{m_2\}\} \qquad 
No\Big(\overline{\semantics{\text{men}}}\Big) = \{\emptyset\}
\]

\noindent
In the first step of computation of the meaning of  ``some men sneeze'', we obtain:

\begin{align*}
(\delta_N \otimes 1_{N})\Big(\overline{\semantics{\text{men}}} \otimes \overline{\semantics{\text{sneeze}}}\Big) =&\\
  \{(\{\emptyset\}, \{\emptyset\}), (\{m_1\}, \{m_1\}), (\{m_2\}, \{m_2\}), (\{m_1, m_2\}, \{m_1, m_2\}) \} \otimes  \{\emptyset, \{m_1\}, \{c_1\}, \{m_1, c_1\}\} 
\end{align*}

\noindent
In the second step, we obtain:
\begin{align*}
\Big(Some \otimes \mu\Big) \Big ( \{(\{\emptyset\}, \{\emptyset\}), (\{m_1\}, \{m_1\}), (\{m_2\}, \{m_2\}), (\{m_1, m_2\}, \{m_1, m_2\}) \} \otimes  \{\emptyset, \{m_1\}, \{c_1\}, \{m_1, c_1\}\}  \Big) =&\\
Some\Big ( \{\{m_1\}, \{m_2\}, \{m_1, m_2\}\} \Big) \otimes \mu \Big( \{\emptyset, \{m_1\}, \{m_2\}, \{m_1, m_2\}\}  \otimes \{\emptyset, \{m_1\}, \{c_1\}, \{m_1, c_1\}\} \Big)=&\\
\{\{m_1\}, \{m_2\}, \{m_1, m_2\}\}  \otimes \{\emptyset, \{m_1\}\}
\end{align*}

\noindent
In the last step, we obtain the following via the  relation $\epsilon \colon N \times N \to \{\star\}$ being $\{((\{m_1\}, \{m_1\}), \star)\}$:

\begin{align*}
\epsilon\Big(\{\{m_1\}, \{m_2\}, \{m_1, m_2\}\}  \otimes \{\emptyset, \{m_1\}\}\Big) = \{\star\}
\end{align*}

\noindent
Hence, the meaning of the sentence is true.  For the sentence ``One man sneezes'',  the second and third steps of computation are as follows:

\begin{align*}
\epsilon\Big( One\Big ( \{\{m_1\}, \{m_2\}, \{m_1, m_2\}\} \Big) \otimes \mu \Big( \{\emptyset, \{m_1\}, \{m_2\}, \{m_1, m_2\}\}  \otimes \{\emptyset, \{m_1\}, \{c_1\}, \{m_1, c_1\}\} \Big)\Big) =&\\
\epsilon \Big(\{\{m_1\}, \{m_2\}\}  \otimes \{\emptyset, \{m_1\}\}\Big) = \{\star\}
\end{align*}

\noindent
So the meaning of this sentence is also true (it has the same $\epsilon$ relation as the previous case). Now consider the case of the  sentence ``no man sneezes'' in which case $No (\semantics{\text man}) = \emptyset$. In this case we obtain  the following at the final step of computation

\begin{align*}
\epsilon\Big( No\Big ( \{\{m_1\}, \{m_2\}, \{m_1, m_2\}\} \Big) \otimes \mu \Big( \{\emptyset, \{m_1\}, \{m_2\}, \{m_1, m_2\}\}  \otimes \{\emptyset, \{m_1\}, \{c_1\}, \{m_1, c_1\}\} \Big)\Big) =&\\
\epsilon \Big(\{\emptyset\}  \otimes \{\emptyset, \{m_1\}\}\Big) = \{\star\}
\end{align*}

\noindent
However, this is not a true sentence, since we have that the second element of the pair in  $\{\emptyset\}  \otimes \{\emptyset, \{m_1\}\}$ is not the singleton set $\{\emptyset\}$,  that is $\{\emptyset, \{m_1\}\} \neq \{\emptyset\}$.  If we had $\overline{\semantics{\text dog}} = \{\emptyset, \{d_1\}\}$, then the compact closed meaning of the sentence ``No dogs sneeze''  would become true, since we would have
\[
\mu(\{\emptyset, \{d_1\}\} \otimes \{\emptyset, \{m_1\}, \{c_1\}, \{m_1, c_1\}\}) = \{\emptyset\}
\]
and also that  $\epsilon (\{\emptyset\}  \otimes \{\emptyset\}) = \{\star\}$.


\section{Truth-Theory in Vector Spaces} 

One can do the same calculations as in $Rel$ in $FVect$ and obtain the same truth theoretic meanings in vector spaces. In this case, all we have to do is to model a set $N = \{n_1, n_2, \cdots\}$, by a vector space $V_N$ spanned by $N$, that is $V_N = \{n_i\}_i$. In the case of our model, $N = {\cal P}(U)$, for $U$ a set of individuals, hence $V_N$ is spanned by subsets of $U$.  Denoting these by $U_i$, we have $V_{{\cal P}(U)} = \{U_i\}_i$. The one element set $\{\star\}$ is then modelled  by the one dimensional vector space. This space models the sentence space, that is we have $S = V_{\{\star\}}$.  

We demonstrate the computation for the truth theoretic meaning of ``Q Sbj Verb'' in the above vector space below. For the meanings of the words therein, we have:

\begin{eqnarray*}
\ov{\text{Sbj}} &=& \sum_i U_i \qquad \text{for} \quad U_i \in {\cal P}(\semantics{\text{Sbj}})\\
\ov{\text{Verb}} &=& \sum_j U_j  \otimes \{\star\} \cong \sum_i U_j  \qquad \text{for} \quad  U_j \in {\cal P}(\semantics{\text{Verb}})\\
Det\Big(\ov{\text{Sbj}}\Big) &=& \sum_k U_k \qquad \text{for} \quad   U_k \in Det(\semantics{\text{Sbj}})
\end{eqnarray*}




The first step of the computation  is as follows:


\begin{align*}
(\delta_N \otimes 1_{N})\Big(\ov{\text Sbj} \otimes \ov{\text Verb}\Big) =   (\delta_N \otimes 1_{N})\Big(\sum_i {U}_i  \otimes \sum_j {U}_j \Big ) = (\sum_i {U}_i \otimes {U}_i) \otimes (\sum_j {U}_j)
\end{align*}

\noindent
In the second step, we obtain:

\begin{align*}
(Det \otimes  \mu_N)\Big(\sum_i {U}_i \otimes {U}_i) \otimes (\sum_j {U}_j\Big) =  Det(\sum_i {U}_i) \otimes  (\sum_i \sigma_{ij} {U}_i)   = \sum_k U_k \otimes \sum_i \sigma_{ij} U_i
\end{align*}

\noindent
The final step  provides us with the following:

\begin{align*}
(\epsilon_{N})  \Big(\sum_k U_k \otimes \sum_i \sigma_{ij} {U}_i  \Big) =  \sum_{ijk}  \langle  U_k   \mid  \sigma_{ij} {U}_i \rangle  
\end{align*}

\begin{definition}
\label{deftrue}
The vector space meaning of the sentence ``Q Sbj Verb'' is true if and only if $Det(\ov{\text Sbj}) \neq 0$ and we have 
\[
\epsilon \circ (Det \otimes \mu) \circ (\sigma \otimes 1_N) \Big(\ov{\text{Sbj}} \otimes \ov{\text{Verb}} \Big) \geqslant 1 
\] 
For the case when  $Det(\ov{\text Sbj}) = 0$,  we have to have the above as well as that the second element of the pair in  the result of  $(1_N \otimes \mu) \circ (\sigma \otimes 1_N) \Big(\ov{\text{Sbj}} \otimes \ov{\text{Verb}} \Big)$ is  the vector with only one non-zero coefficient on the basis $\emptyset$. 
\end{definition}

\begin{proposition}
The vector space meaning of a quantified sentence  is true if and only if  its relational meaning is true.
\end{proposition}

\bigskip
\noindent
{\bf Example.}
As an example,  consider  the  meaning  of  ``some men sneeze'', in the first step of the computation we have:
\begin{eqnarray*}
&&(\delta \otimes 1_N) \Big(\ov{\text men} \otimes \ov{\text sneeze}\Big) = (\delta \otimes 1) \Big((\emptyset + \{m_1\} + \{m_2\} + \{m_1, m_2\}) \otimes (\emptyset + \{m_1\} + \{c_1\} + \{m_1, c_1\}) \Big)\\
&&=  (\emptyset \otimes \emptyset + \{m_1\} \otimes \{m_1\} + \{m_2\} \otimes \{m_2\} + \{m_1, m_2\} \otimes \{m_1, m_2\}) \otimes (\emptyset + \{m_1\} + \{c_1\} + \{m_1, c_1\})
\end{eqnarray*}

\noindent
In the second step, we apply $(Some \otimes \mu)$ to the above and obtain:
\begin{eqnarray*}
&& Some\Big(\emptyset + \{m_1\} + \{m_2\} + \{m_1, m_2\}\Big) \otimes \mu \Big((\emptyset  + \{m_1\}  + \{m_2\}  + \{m_1, m_2\}) \otimes (\emptyset + \{m_1\} + \{c_1\} + \{m_1, c_1\})\Big) \\
&&= (\{m_1\} + \{m_2\} + \{m_1, m_2\}) \otimes (\emptyset + \{m_1\})
\end{eqnarray*}

\noindent
In the final step, we apply $\epsilon$ to the above and obtain:
\[
\langle \{m_1\} + \{m_2\} + \{m_1, m_2\} \mid \emptyset + \{m_1\} \rangle =  1
\]

\noindent
For ``one man sneezes'' the hole computation is as follows:
\begin{eqnarray*}
&&\epsilon(One \otimes \mu)(\delta \otimes 1_N) \Big(\ov{\text men} \otimes \ov{\text sneeze}\Big) = \\
&&  \epsilon(One \otimes \mu)(\emptyset \otimes \emptyset + \{m_1\} \otimes \{m_1\} + \{m_2\} \otimes \{m_2\} + \{m_1, m_2\} \otimes \{m_1, m_2\}) \otimes (\emptyset + \{m_1\} + \{c_1\} + \{m_1, c_1\})\\
&&= \epsilon \Big (One\big(\emptyset + \{m_1\} + \{m_2\} + \{m_1, m_2\}\Big) \otimes \mu \Big((\emptyset  + \{m_1\}  + \{m_2\}  + \{m_1, m_2\}) \otimes (\emptyset + \{m_1\} + \{c_1\} + \{m_1, c_1\})\Big) \\
&&= \epsilon \Big((\{m_1\} + \{m_2\} ) \otimes (\emptyset + \{m_1\})\Big)\\
&&= \langle \{m_1\} + \{m_2\}  \mid \emptyset + \{m_1\} \rangle =  1
\end{eqnarray*}


\noindent
For ``no man sneezes'', we have that $Det(\ov{\text Sbj}) = 0$, hence for this to be true its $FVect$ meaning should be above 1 and also $\mu \Big(\ov{\text{Sbj}} \otimes \ov{\text{Verb}} \Big)$ should be the $\emptyset$ vector. 

The first step of the computation  of the $FVect$ meaning of the sentence is as before, in the second and third steps we obtain:

\begin{eqnarray*}
&&\epsilon \Big (No (\emptyset + \{m_1\} + \{m_2\} + \{m_1, m_2\}) \otimes  \mu\big(\emptyset  + \{m_1\}  + \{m_2\}  + \{m_1, m_2\}) \otimes (\emptyset + \{m_1\} + \{c_1\} + \{m_1, c_1\}\big)\Big)\\
&&= 
\epsilon \Big(\{\emptyset\} \otimes (\emptyset + \{m_1\}) \Big) = 1
\end{eqnarray*}

\noindent
But this sentence is not true, since we have 
\[
(1_N \otimes \mu) \circ (\sigma \otimes 1_N)(\ov{\text men} \otimes \ov{\text sneeze})   = \emptyset + \{m_1\} \neq \emptyset
\]
Whereas, for  ``no dogs sneeze'', where $\ov{\text dogs} = \emptyset + \{d_1\}$, we will have:
\[
\mu\big(\ov{\text dogs} \otimes \ov{\text sneeze}\big) = \mu \big((\emptyset + \{d_1\}) \otimes (\emptyset + \{m_1\} + \{c_1\} + \{m_1, c_1\})\big) = \emptyset
\]
And also that 
\[\epsilon(No \otimes \mu) (\delta \otimes 1)(\ov{\text dogs} \otimes \ov{\text sneeze}) =
\epsilon(\emptyset \otimes \emptyset) = 1
\]

\noindent
Hence ``no dogs sneeze'' is true. 

%%% Local Variables: 
%%% mode: latex
%%% TeX-master: "Quant"
%%% End: 
