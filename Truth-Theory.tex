\section{Truth Theoretic Interpretation}

Given the set-theoretical model $(U, \semantics{\ })$ of a language ${\cal L}_{\Sigma}$, a concrete relational instantiation of the abstract compact closed categorical interpretation is  provided by the tuple $({\cal C}_{{\cal P}{\cal P}(U), \{\star\}}, \overline{\semantics{\ }})$, defined as follows: 
\begin{itemize}
\item For a word with a lexical category N,NP, and VP, that is,  for $s \in \{$N, NP, VP$\}$ and any  $w \in \Sigma$ such that  $(w,s) \in {\cal X}_{\Sigma}$,  we have  
\[
\overline{\semantics{w}} \in {\cal P}{\cal P}(U)
\]
represented by the morphism $\{\star\} \to {\cal P}{\cal P}(U)$. 
\item For words with lexical category VP, we have
\item For words with lexical category V, we have
\[
\overline{\semantics{w}} \in   {\cal P}{\cal P}(U) \times  {\cal P}{\cal P}(U) 
\]
Since in $Rel$, the tensor product is the cartesian product, the above is an element of ${\cal P}{\cal P}(U) \otimes {\cal P}{\cal P}(U)$, represented by the morphism $\{\star\} \to {\cal P}{\cal P}(U) \otimes {\cal P}{\cal P}(U)$. 

\item For a word with the lexical category Det, that is a $d \in \Sigma$ such that $(d, Det) \in {\cal X}$ and a $w \in \Sigma$ such that $(w, \text{N}) \in {\cal X}$, we have 
\[
\overline{\semantics{d}}  (\overline{\semantics{w}}) \subseteq {\cal P}{\cal P}(U) \times {\cal P}{\cal P}(U)
\]
This is a relation on the powerset of powerset of $U$; it encodes the generalised quantifier map $\semantics{d}$ in the form of a relation.  
\end{itemize}

In this instantiation,   the sentence space is the unit of the tensor,   hence meaning of a sentence with a quantified phrase at its subject position simplifies as follows:
\[
\epsilon_{{\cal P}{\cal P}(U)}(\overline{\semantics{d}} \otimes  \mu_{{\cal P}{\cal P}(U)})(\Delta_{{\cal P}{\cal P}(U)} \otimes 1_{{{\cal P}{\cal P}(U)}} \otimes \epsilon_{{\cal P}{\cal P}(U)}) \circ
(\overline{\semantics{n}} \otimes \overline{\semantics{{v}}} \otimes \overline{\semantics{np}})
\]
Similarly, meaning of a sentence with a quantified phrase at its object position simplifies as follows:
\[
\epsilon_{{\cal P}{\cal P}(U)} \circ (\mu_{{\cal P}{\cal P}(U)} \otimes \overline{\semantics{d}}) \circ (\epsilon_{{\cal P}{\cal P}(U)} \otimes 1_{{{\cal P}{\cal P}(U)}} \otimes \Delta_{{\cal P}{\cal P}(U)}) \circ (\overline{\semantics{np}} \otimes \overline{\semantics{v}} \otimes \overline{\semantics{n}})
\]


\bigskip
\noindent
{\bf Discussion about the choice of ${\cal P}{\cal P}(U)$ as the atomic object}. The reason one should not take $U$ to be the atomic object is that then one has to work with morphisms of the type $I \to U$, which denote elements of $U$, whereas meanings of words are subsets of $U$. The choice of ${\cal P}(U)$ fails because although now we have morphisms of the type $I \to {\cal P}(U)$, which do denote subsets of $U$, the application of the Frobenius $\mu \colon {\cal P}(U) \times {\cal P}(U)$ on them will yield unwanted results, for instance for two subsets $\{m_1, m_2\}$ and $\{m_2, m_3\}$, one wants $\mu$ to return their intersection that is $\{m_3\}$, whereas here it will return the empty set. The choice of ${\cal P}{\cal P}(U)$ is the right choice here, as shown below in the examples and corresponding propositions.   




\bigskip
\bigskip
\noindent
{\bf Example (I): Intransitive Verb.} As a truth-theoretic  example,  suppose $U = \{m_1, m_2, c_1\}$, from which we have two male individuals $m_1, m_2$  and a cat   individual $c_1$.  Suppose further that  the verb `sneeze'  applies to individuals $m_1$ and $c_1$. Consider the following embedding:

\[
\overline{\semantics{\text{men}}} =  \downarrow_{\neq \emptyset}\{m_1, m_2\}  \qquad
\overline{\semantics{\text{cat}}} =  \downarrow_{\neq \emptyset}\{c_1\}  \qquad
\overline{\semantics{\text{sneeze}}} = \downarrow_{\neq \emptyset}\{m_1, c_1\}
\]
So we have:
\[
\overline{\semantics{\text{men}}} =  \{\{m_1\}, \{m_2\}, \{m_1, m_2\}\}  \qquad
\overline{\semantics{\text{cat}}} =  \{\{c_1\}\}  \qquad
\overline{\semantics{\text{sneeze}}} = \{\{m_1\}, \{c_1\}, \{m_1, c_1\}\}
\]

\noindent
For  quantifiers, we set:
\[
\overline{\semantics{d}}  (\overline{\semantics{w}})  :=  \semantics{d}(\semantics{w}) 
\]
As examples of quantified phrases we have:

\[
Some\Big({\semantics{\text{men}}}\Big) =  \{\{m_1\}, \{m_2\}, \{m_1, m_2\}\} \qquad
All\Big({\semantics{\text{man}}}\Big) = \{\{m_1, m_2\}\} \qquad 
No\Big({\semantics{\text{man}}}\Big) = \{\emptyset\} 
\]

\noindent
The goal is to compute the meaning of  ``some men sneeze''. In the first step of computation we obtain (the subscripts are dropped,  they are always ${\cal P}{\cal P}(U)$):

\begin{align*}
(\Delta \otimes 1)\Big(\overline{\semantics{\text{men}}} \otimes \overline{\semantics{\text{sneeze}}}\Big) =&\\
  \{(\{m_1\}, \{m_1\}), (\{m_2\}, \{m_2\}), (\{m_1, m_2\}, \{m_1, m_2\}) \} \otimes  \{\{m_1\}, \{c_1\}, \{m_1, c_1\}\} 
\end{align*}

\noindent
In the second step, we obtain:
\begin{align*}
&\Big(\overline{\semantics{Some}} \otimes \mu\Big) \Big ( \{(\{m_1\}, \{m_1\}), (\{m_2\}, \{m_2\}), (\{m_1, m_2\}, \{m_1, m_2\}) \} \otimes  \{\{m_1\}, \{c_1\}, \{m_1, c_1\}\}  \Big) =\\
&\overline{\semantics{Some}}\Big (\{\{m_1\}, \{m_2\}, \{m_1, m_2\}\} \Big) \otimes \mu \Big( \{\{m_1\}, \{m_2\}, \{m_1, m_2\}\}  \otimes \{\{m_1\}, \{c_1\}, \{m_1, c_1\}\} \Big)=\\
&{\semantics{Some}}\Big (\{m_1, m_2\} \Big) \otimes\{\{m_1\}\}=\\
&\{\{m_1\}, \{m_2\}, \{m_1, m_2\}\}  \otimes \{\{m_1\}\}
\end{align*}

\noindent
In the last step, we obtain the following:

\begin{align*}
\epsilon\Big(\{\{m_1\}, \{m_2\}, \{m_1, m_2\}\}  \otimes \{\{m_1\}\}\Big) = \{\star\}
\end{align*}

\noindent
Hence, the meaning of the sentence is true.  For the sentence ``all men sneeze'',  one applies $(\overline{\semantics{all}} \otimes \mu)$ to the result of the first step as above. The second and third steps of computation are as follows:

\begin{align*}
&\epsilon   \Big(\overline{\semantics{all}}\Big ( \{\{m_1\}, \{m_2\}, \{m_1, m_2\}\} \Big) \otimes \mu \Big( \{\{m_1\}, \{m_2\}, \{m_1, m_2\}\}  \otimes \{\{m_1\}, \{c_1\}, \{m_1, c_1\}\} \Big)  \Big) =\\
&\epsilon \Big(\{\{m_1, m_2\}\}  \otimes \{\{m_1\}\}\Big) = \emptyset
\end{align*}

\noindent
So the meaning of this sentence is false.   For the sentence of `No men sneeze',  we have the following:

\begin{align*}
&\epsilon   \Big(\overline{\semantics{no}}\Big ( \{\{m_1\}, \{m_2\}, \{m_1, m_2\}\} \Big) \otimes \mu \Big( \{\{m_1\}, \{m_2\}, \{m_1, m_2\}\}  \otimes \{\{m_1\}, \{c_1\}, \{m_1, c_1\}\} \Big)  \Big) =\\
&\epsilon \Big(\{\emptyset\}  \otimes \{\{m_1\}\}\Big) = \emptyset
\end{align*}


\bigskip
\noindent
{\bf Example (II): Transitive Verb.} Suppose both of  the male individuals love the cat. We set:
\[
\overline{\semantics{\text{love}}} := \downarrow_{\neq \emptyset} Dom(\text{love}) \times \downarrow_{\neq \emptyset} Codom(\text{love})
\] 
Hence we have:
\[
\overline{\semantics{\text{love}}} := \{\{m_1\}, \{m_2\}, \{m_1, m_2\}\} \times \{\{c_1\}\}
\]
Meaning of the sentence `Some men love cats' is computed as follows. In the first step we compute:

\begin{align*}
(\sigma \otimes 1 \otimes \epsilon)(\overline{\semantics{\text{men}}} \otimes \overline{\semantics{\text{love}}} \otimes \overline{\semantics{\text{cats}}}) =& \\
   \{(\{m_1\}, \{m_1\}), (\{m_2\}, \{m_2\}), (\{m_1, m_2\}, \{m_1, m_2\}) \} \otimes
   \{\{m_1\}, \{m_2\}, \{m_1, m_2\}\} \otimes \epsilon(\{\{c_1\}\} \otimes \{\{c_1\}\}) = \\
     \{(\{m_1\}, \{m_1\}), (\{m_2\}, \{m_2\}), (\{m_1, m_2\}, \{m_1, m_2\}) \} \otimes
   \{\{m_1\}, \{m_2\}, \{m_1, m_2\}\} \otimes  \{\star\}  = \\
     \{(\{m_1\}, \{m_1\}), (\{m_2\}, \{m_2\}), (\{m_1, m_2\}, \{m_1, m_2\}) \} \otimes
   \{\{m_1\}, \{m_2\}, \{m_1, m_2\}\}  
\end{align*}

\noindent
In the second step, we apply $\overline{\semantics{\text Some}} \otimes \mu$ to the above and compute:

\begin{align*}
\overline{\semantics{\text Some}} (\{\{m_1\}, \{m_2\}, \{m_1, m_2\}\}) \otimes \mu(\{\{m_1\}, \{m_2\}, \{m_1, m_2\}\} \otimes \{\{m_1\}, \{m_2\}, \{m_1, m_2\}\}) = \\
 \semantics{\text Some}(\{m_1, m_2\})  \otimes \{\{m_1\}, \{m_2\}, \{m_1, m_2\}\} = \\
\{\{m_1\}, \{m_2\}, \{m_1,m_2\}\} \otimes   \{\{m_1\}, \{m_2\}, \{m_1, m_2\}\}\\
\end{align*}

\noindent
In the final step, we apply $\epsilon$ to the above and compute:

\[
\epsilon(\{\{m_1\}, \{m_2\}, \{m_1,m_2\}\} \otimes   \{\{m_1\}, \{m_2\}, \{m_1, m_2\}\}) = \{\star\}
\]
So the sentence is true. For ``all men love cats'', the first step of the computation is as above. For the second and third steps we compute:
\begin{align*}
\epsilon\Big(\overline{\semantics{all}} \big(\{\{m_1\}, \{m_2\}, \{m_1, m_2\}\}\big) \otimes  \mu \big(\{\{m_1\}, \{m_2\}, \{m_1, m_2\}\} \otimes \{\{m_1\}, \{m_2\}, \{m_1, m_2\}\}\big)\Big)=\\
\epsilon(\{\{m_1, m_2\}\} \otimes  \{\{m_1\}, \{m_2\}, \{m_1, m_2\}\}) = \{\star\}
\end{align*}

\noindent
So this sentence is also true. Whereas for ``no men love cats'' we have:
\begin{align*}
\epsilon\Big(\overline{\semantics{no}} \big(\{\{m_1\}, \{m_2\}, \{m_1, m_2\}\}\big) \otimes  \mu \big(\{\{m_1\}, \{m_2\}, \{m_1, m_2\}\} \otimes \{\{m_1\}, \{m_2\}, \{m_1, m_2\}\}\big)\Big)=\\
\epsilon(\{\emptyset\} \otimes  \{\{m_1\}, \{m_2\}, \{m_1, m_2\}\}) =\emptyset
\end{align*}

\noindent
So the meaning of this sentence is also false.  







%%% Local Variables: 
%%% mode: latex
%%% TeX-master: "Quant"
%%% End: 
