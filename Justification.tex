\section{Justification}
\label{just}

%%% Local Variables: 
%%% mode: latex
%%% TeX-master: "Quant"
%%% End: 


Recalling that, as shown in \cite{CoeckePaquettePavlovic09,CoeckePaq},  the Frobenius $ \mu$ map is the analog of  set-theoretic intersection and the compact closed  epsilon map is the analog of  set-theoretic application, it is not hard to show that  the truth-theoretic interpretation of the compact closed semantics of quantified sentences provides us with the same meaning as their generalised quantifier semantics. In what follows,  we make this formal as follows. 

\begin{definition}
The meaning of a sentence in a concrete relational instantiation of the compact closed categorical interpretation is true  iff $\epsilon_{{\cal P}{\cal P}(U)} (\overline{\semantics{np}} \otimes \overline{\semantics{vp}}) = \{\star\}$ and  is false otherwise. 
\end{definition}

\begin{proof}
That the interpretation of atoms is ??? and that the rules preserve truth?
\end{proof}

\begin{lemma}
Using the non-empty down set embedding, the meaning of a sentence with a quantified phrase at its subject position becomes equivalent to the following
\[
 \{\star \mid  D_k = A_i = B_j , D_k \in \semantics{d}(\semantics{\text{N}}),  A_i \in {\cal P}_{\neq \emptyset} (\semantics{\text{N}}), B_j \in  {\cal P}_{\neq \emptyset} (\semantics{\text{VP}})\}
 \]
\end{lemma}

\begin{proof}
The non-empty down set embedding means that we have the following: 

\begin{eqnarray*}
\overline{\semantics{{n}}} &=& \{A_i \mid A_i \in {\cal P}_{\neq \emptyset}(\semantics{{n}})\} \qquad 
\overline{\semantics{d}}\Big(\overline{\semantics{{n}}}\Big) = \{D_o \mid D_o \in {\semantics{d}}(\semantics{{n}})\}\\
\overline{\semantics{{np}}} &=& \{C_l \mid C_l \in {\cal P}_{\neq \emptyset}(\semantics{{np}})\} \qquad 
\overline{\semantics{{v}}} = \{(B_j, B_k) \mid B_j, B_k  \in {\cal P}_{\neq \emptyset}(\semantics{{v}})\}\\
\end{eqnarray*}

\noindent
The meaning of a sentence with a quantified subject is computed in three steps. In the  first step, we obtain:

\begin{align*}
(\delta_N \otimes 1_{N})\Big(\overline{\semantics{\text{N}}} \otimes \overline{\semantics{\text{VP}}}\Big) =  \{(A_i, A_i) \mid A_i \in {\cal P} (\semantics{\text{N}})\} \otimes \{B_j \mid B_j \in  {\cal P}_{\neq \emptyset} (\semantics{\text{VP}})\}
\end{align*}


\noindent
In the second step, we obtain:

\begin{align*}
(Det \otimes  \mu_N)\Big(\{(A_i, A_i) \mid A_i \in {\cal P}_{\neq \emptyset} (\semantics{\text{Sbj}})\} \otimes \{B_j \mid B_j \in  {\cal P} (\semantics{\text{Verb}})\}\Big) &=\\
Det\Big(\{A_i \mid A_i \in {\cal P}_{\neq \emptyset} (\semantics{\text{Sbj}})\}\Big) \otimes \{A_i \mid A_i = B_j, A_i \in {\cal P} (\semantics{\text{Sbj}}), B_j \in  {\cal P}(\semantics{\text{Verb}})\} &=\\
\{D_k \mid D_k \in Det(\semantics{\text{N}})\} \otimes \{A_i \mid A_i = B_j, A_i \in {\cal P}_{\neq \emptyset} (\semantics{\text{N}}), B_j \in  {\cal P}(\semantics{\text{VP}})\}
\end{align*}

\noindent
In the final step, we obtain:
\begin{align*}
\epsilon\Big(\{D_k \mid D_k \in Det(\semantics{\text{Sbj}})\} \otimes \{A_i \mid A_i = B_j, A_i \in {\cal P} (\semantics{\text{Sbj}}), B_j \in  {\cal P}(\semantics{\text{Verb}})\}\Big) &=\\
 \{\star \mid  D_k = A_i, D_k \in Det(\semantics{\text{N}}), A_i = B_j, A_i \in {\cal P}_{\neq \emptyset} (\semantics{\text{N}}), B_j \in  {\cal P} (\semantics{\text{VP}})\}
\end{align*}
\end{proof}





\begin{definition}
\label{deftrue}
The compact closed meaning of the sentence ``Det N VP'' is true  in $Rel$ if and only if 
\[
\epsilon \circ (Det \otimes \mu) \circ (\sigma \otimes 1_N) \Big(\overline{\semantics{\text{N}}} \otimes \overline{\semantics{\text{VP}}} \Big) = \{\star\} 
\] 
\end{definition}


\begin{proposition}
The  compact closed meaning of a quantified sentence computed in $Rel$ is true if and only if  its generalised quantifier meaning is true.
\end{proposition}

\begin{proof}
For the right to left direction,  suppose  the generalised quantifier meaning of the sentence is true, that is $\semantics{\text{Sbj}} \cap \semantics{\text{Verb}} \in Det(\semantics{\text{Sbj}}$ and consider the case when  $Det(\semantics{\text Sbj}) \neq \{\emptyset\}$. We have to show that $\epsilon \circ (Det \otimes \mu) \circ (\sigma \otimes 1_N) \Big(\overline{\semantics{\text{Sbj}}} \otimes \overline{\semantics{\text{Verb}}} \Big) = \{\star\}$.  For this, we need to show that there is a set $G$ equal to $D_k= A_i = B_j$ such that $G$ is in $Det(\semantics{\text{Sbj}})$ and $ {\cal P} (\semantics{\text{Sbj}})$ and  ${\cal P} (\semantics{\text{Verb}})$.  Take $G$ to be $\semantics{\text{Sbj}} \cap \semantics{\text{Verb}}$.  Then, it is a subset of   $\semantics{\text{Sbj}}$, hence an element of  $ {\cal P} (\semantics{\text{Sbj}})$, a subset of  $ \semantics{\text{Verb}}$, hence an element of $ {\cal P} (\semantics{\text{Verb}})$, and an element of $Det(\semantics{\text{Sbj}}$. 


For the left to right direction, suppose the compact closed meaning of the sentence is true in $Rel$, and consider  the case when $Det(\semantics{\text Sbj}) \neq \{\emptyset\}$. Then  we have  subsets $D_k = A_i = B_j$ such that $D_k \in Det(\semantics{\text Sbj}), A_i \in {\cal P}(\semantics{\text Sbj})$, and $B_j \in {\cal P}(\semantics{\text Verb})$.  Pick an arbitrary such subset, e.g.  $G$, such that it   equal to  $D_k = A_i = B_j$; we have that $G \in   {\cal P}(\semantics{\text Sbj})$, hence $G \subseteq \semantics{\text Sbj}$; and that $G \in  {\cal P}(\semantics{\text Verb})$, hence $G \subseteq \semantics{\text Verb}$. It thus follows that $G \subseteq \semantics{\text Sbj}  \cap \semantics{\text Verb}$. At the same time $G \in Det(\semantics{\text Sbj})$, hence the generalised quantifier meaning is also true.  


\end{proof}
