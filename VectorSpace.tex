\section{Vector Space  Interpretation}

%%% Local Variables: 
%%% mode: latex
%%% TeX-master: "Quant"
%%% End: 

Given the set-theoretical model $(U, \semantics{\ })$ of a language ${\cal L}_{\Sigma}$, a   vector instantiation of the abstract compact closed categorical interpretation is  provided by the tuple $({\cal C}, W,S, \ov{\text{\ }})$, where $W$ is a vector space with a basis $\{n_i\}_i$ and $S$ is a vector space with a basis $\{s_j\}_j$.  Vector meanings of words are as follows (the corresponding morphisms are obtained as before): 

\begin{itemize}
\item For a word w with a lexical category N, NP we have  
\[
 \ov{\text{w}} \colon I \to W \qquad \mbox{given by} \quad  \sum_i C_i \ov{n}_i
\]
\item For words w with lexical category VP, we have
\[
 \ov{\text{w}}  \colon I \to W \otimes S \qquad \mbox{given by} \quad  \sum_{ij} C_{ij} \ov{n}_i \otimes \ov{s}_j 
\]
\item For words w with lexical category V, we have
\[
\ov{\text{w}}  \colon I \to W \otimes S \otimes W \qquad \mbox{given by} \quad   \sum_{ijk} C_{ijk} \ov{n}_i \otimes \ov{s}_j \otimes \ov{n}_k
\]
\item For a word d with the lexical category Det and w a word with the lexical category N, we have 
\[
 d \colon W \to W \qquad \mbox{given by} \quad d(\ov{\text {w}})  = \sum_o C_o \ov{n}_o \quad \text{for} \quad \ov{\text{w}} \in W
\]
Supposing $\ov{\text{w}} =  \sum_i  C_i \ov{n}_i$, by linearity of $d$, from the above it follows that  $d(\ov{\text {w}})  =  d(\sum_i  C_i \ov{n}_i) = \sum_i C_i d(\ov{n}_i)$.  If we take $d(\ov{n}_i) = \sum_t C^i_t \ov{n}_t$, we obtain  that $\sum_{it} C_i C^i_t \ov{n}_t =  \sum_o C_o \ov{n}_o$. 
\end{itemize}

\noindent
In this instantiation, the  meaning of  a sentence with a quantified subject  is obtained by computing the following:

\[
(\epsilon_W \otimes 1_S) \circ (d \otimes  \mu_W \otimes 1_S) \circ (\Delta_W \otimes 1_{W \otimes S} \otimes \epsilon_W)  \circ (\ov{{n}} \otimes \ov{{v}} \otimes \ov{{np}})
\]
Setting $\ov{{n}} = \sum_l C_l \ov{n}_l, \ov{{v}} = \sum_{ijk} C_{ijk} \ov{n}_i \otimes \ov{s}_j \otimes \ov{n}_k$, and $\ov{{np}} = \sum_r C_r \ov{n}_r$, and   unfolding the morphisms, in the first step of the computation we obtain the following (where $\delta_{rk}$ is 1 when $r=k$ and 0 otherwise):

\begin{align*}
(\Delta_W \otimes 1_{W \otimes S} \circ \epsilon_W)\Big(\sum_l C_l \ov{n}_l \otimes \sum_{ijk} C_{ijk} \ov{n}_i \otimes \ov{s}_j \otimes \ov{n}_k \ \otimes \sum_r C_r \ov{n}_r \Big) = \\
(\sum_l C_l \ov{n}_l \otimes \ov{n}_l) \otimes  (\sum_{ijkr} C_{ijk} C_r \ov{n}_i \otimes \ov{s}_j \otimes \langle \ov{n}_k \mid \ov{n}_r \rangle)=\\
(\sum_l C_l \ov{n}_l \otimes \ov{n}_l) \otimes  (\sum_{ijkr} C_{ijk} C_r \ov{n}_i \otimes \ov{s}_j  \delta_{rk})
\end{align*}

\noindent
In the second step we obtain: 

\begin{align*}
(d \otimes  \mu_W \otimes 1_S)\Big((\sum_l C_l \ov{n}_l \otimes \ov{n}_l) \otimes  (\sum_{ijkr} C_{ijk} C_r \ov{n}_i \otimes \ov{s}_j  \delta_{rk})\Big) =\\ 
\sum_{ijkrl} C_{ijk}  C_r  C_l d(\ov{n_l}) \otimes \mu(\ov{n}_l \otimes \ov{n}_i) \otimes \ov{s}_j \delta_{rk}=\\
\sum_{ijkrl} C_{ijk}  C_r  C_l d(\ov{n_l}) \otimes \delta_{li}\ov{n}_i \otimes \ov{s}_j \delta_{rk}
\end{align*}

\noindent
The final step is as follows:

\begin{align*}
(\epsilon_{W} \otimes 1_S) \Big( \sum_{ijkrl} C_{ijk}  C_r  C_l d(\ov{n_l}) \otimes \delta_{li}\ov{n}_i \otimes \ov{s}_j \delta_{rk} \Big) =
\sum_{ijkrl} C_{ijk}  C_r  C_l \langle d(\ov{n_l}) \mid \ov{n}_i \rangle  \delta_{li} \ov{s}_j \delta_{rk}
\end{align*}

\noindent Now if we instantiate $d(\ov{n}_l)$ to $\sum_t C^i_t \ov{n}_t$, the above further simplifies to the following:
 
 \[
\sum_{ijkrl} C_{ijk}  C_r  C_l \langle \sum_t C^i_t \ov{n}_t \mid \ov{n}_i \rangle  \delta_{li} \ov{s}_j \delta_{rk}=
\sum_{ijkrlt} C_{ijk}  C_r  C_l C^i_t \delta_{ti}  \delta_{li} \ov{s}_j \delta_{rk}
\]

\noindent
Similar computations provide us with the following for the meaning of a sentence with a quantified object:
\[
(1_S \otimes \epsilon_W) \circ (1_S \otimes \mu_W \otimes d) \circ (\epsilon_W \otimes 1_{S \otimes W} \otimes \Delta_W)(\ov{{n}} \otimes \ov{{v}} \otimes \ov{{np}}) = \sum_{ijkrlt} C_{ijk} C_r C_l C^i_t \delta_{ri} \ov{s}_j \delta_{kl} \delta_{kt}
\]

\noindent
{\bf Intransitive Example.} 
As an example, take $W$ to be the two dimensional space with the basis $\{\ov{n}_1, \ov{n}_2\}$ and $S$ to be the two dimensional space with the basis $\{\ov{s}_1, \ov{s}_2\}$.  Consider an intransitive sentence with a quantified subject and the following linear expansions for its subject and verb:
\[
\ov{np} := C_1 \ov{n}_1 + C_2 \ov{n}_2
\qquad 
 \ov{vp} := C_{11} \ov{n}_1 \otimes \ov{s}_1 + C_{12} \ov{n}_1 \otimes \ov{s}_2 +  C_{21}  \ov{n}_2 \otimes \ov{s}_1 + C_{22} \ov{n}_2 \otimes \ov{s}_2
 \]
    Suppose further the following for the interpretation of the determiner:
 \begin{equation*}\label{eqDetLin}
d(\ov{{np}}) = d(C_1 \ov{n}_1 + C_2 \ov{n}_2) =  C_1 d(\ov{n}_1) + C_2 d(\ov{n}_2) =  C'_1 \ov{n}_1 + C'_2 \ov{n}_2
\end{equation*}
where we further assume the following for the effect of $d$ on each basis:
\[
d(\ov{n}_1) = C^1_1 \ov{n}_1 + C^1_2 \ov{n}_2 \qquad
d(\ov{n}_2) = C^2_1 \ov{n}_1 + C^2_2 \ov{n}_2 
\]
So we obtain the following equivalence between the application of $d$ on words and on basis vectors:
\[
d(\ov{{np}}) = C'_1 \ov{n}_1 + C'_2 \ov{n}_2 \ = \ (C_1 C_1^1 + C_2 C_1^2) \ov{n}_1 + (C_1C_2^1 + C_2C_2^2) \ov{n}_2
\]

In the first step of the computation of the meaning vector of the sentence `Q np vp' we have:
\begin{align*}
(\Delta_W \otimes 1_{W \otimes S})\Big((C_1\ov{n}_1 + C_2 \ov{n}_2) \otimes (C_{11} \ov{n}_1 \otimes \ov{s}_1 + C_{12} \ov{n}_1 \otimes \ov{s}_2
+ C_{21} \ov{n}_2 \otimes \ov{s}_1 + C_{22} \ov{n}_2 \otimes \ov{s}_2) \Big)=\\
(C_1 \ov{n}_1 \otimes \ov{n}_1 + C_2 \ov{n}_2 \otimes \ov{n}_2) \otimes (C_{11} \ov{n}_1 \otimes \ov{s}_1 + C_{12} \ov{n}_1 \otimes \ov{s}_2
+ C_{21} \ov{n}_2 \otimes \ov{s}_1 + C_{22} \ov{n}_2 \otimes \ov{s}_2)
\end{align*}
In the second step of computation, we apply $(d \otimes \mu_W \otimes 1_S)$ to the above and obtain:
\[
(C_1 d(\ov{n}_1) + C_2 d(\ov{n}_2)) \otimes (C_1 C_{11} \ov{n}_1 \otimes \ov{s}_1 +  C_1 C_{12} \ov{n}_1 \otimes \ov{s}_2 + C_2 C_{21} \ov{n}_2 \otimes \ov{s}_1 + C_2 C_{22} \ov{n}_2 \otimes \ov{s}_2) \qquad (*)
\]
In the final step, we apply $(\epsilon_W \otimes 1_S)$ to the above and obtain:
\begin{eqnarray*}
C_1 C_1 C_{11}  \langle d(\ov{n}_1) \mid \ov{n}_1\rangle   \ov{s}_1 +  C_1 C_1 C_{12} \langle d(\ov{n}_1) \mid \ov{n}_1 \rangle  \ov{s}_2 + C_1 C_2 C_{21} \langle d(\ov{n}_1) \mid \ov{n}_2\rangle  \ov{s}_1 + C_1 C_2 C_{22} \langle d(\ov{n}_1) \mid  \ov{n}_2 \rangle  \ov{s}_2 \\
+\\
C_2 C_1 C_{11}  \langle d(\ov{n}_2) \mid \ov{n}_1\rangle   \ov{s}_1 +  C_2 C_1 C_{12} \langle d(\ov{n}_2) \mid \ov{n}_1 \rangle  \ov{s}_2 + C_2 C_2 C_{21} \langle d(\ov{n}_2) \mid \ov{n}_2\rangle  \ov{s}_1 + C_2 C_2 C_{22} \langle d(\ov{n}_2) \mid  \ov{n}_2 \rangle  \ov{s}_2
\end{eqnarray*}
Now,  using the linear expansion of $d$, the above is further simplified   to the following:
\begin{eqnarray*}
C_1 C_1 C_{11}C_1^1 \ov{s}_1 + C_1 C_1 C_{12} C_1^1 \ov{s}_2 + C_1 C_2 C_{21} C_2^1 \ov{s}_1 + C_1 C_2 C_{22} C_2^1 \ov{s}_2\\
+\\
C_2C_1C_{11} C_1^2\ov{s}_1 + C_2C_1C_{12} C_1^2 \ov{s}_2 + C_2C_2C_{21} C_2^2 \ov{s}_1 +
C_2 C_2 C_{22} C_2^2 \ov{s}_2
\end{eqnarray*}

To obtain a more readable notation,  instead of expanding, which is what we have been doing so far, let us factor things out a bit.  First note that the $(*)$ above is equivalent to the following by linearity of the map $d$:
\[
d(C_1 \ov{n}_1 + C_2 \ov{n}_2) \otimes (C_1 C_{11} \ov{n}_1 \otimes \ov{s}_1 +  C_1 C_{12} \ov{n}_1 \otimes \ov{s}_2 + C_2 C_{21} \ov{n}_2 \otimes \ov{s}_1 + C_2 C_{22} \ov{n}_2 \otimes \ov{s}_2) 
\]
Then, using a matrix notation where we assume column vectors are elements of $W$ and 2 by 2 matrices elements of $W \otimes S$, observe that the above can be written down as follows:
\[
\left(
 \left ( \begin{array}{cc}
 C_1 & C_1\\ C_2 & C_2
  \end{array} \right)
 \ \odot \ 
 \left ( \begin{array}{cc}
 C_{11} & C_{12}\\ C_{21} & C_{22}
\end{array} \right ) \right)
 \quad \times \quad
 d\left ( \begin{array}{c} C_1 \\ C_2 \end{array} \right )
 \hspace{2cm} (**)
\]
Here the map $d$ is being applied to the vector meaning of the word $np$  rather to the basis vectors of the vector space $W$. In the preceding computations, the map $d$ was being applied to the basis vectors and the result of the final step of the computation was expressed in that form.  A routine computations shows that the equivalence between the above assumptions on the applications of the $d$, on words or basis vectors,   in linear expansion form has the following matrix form:

\[
d\left ( \begin{array}{c} C_1 \\ C_2 \end{array} \right ) \ = \ \left ( \begin{array}{c} C'_1 \\ C'_2 \end{array} \right ) \ = \ 
\left( \begin{array}{c}
C_1C_1^1 + C_2 C_1^2\\
C_1C_2^1 + C_2C_2^2
\end{array}\right)
\]
This can then be replaces and used in the $(**)$ formula. 


\smallskip
\noindent
{\bf Transitive Example.}
Consider the same two dimensional $W$ and $S$ spaces for the case of transitive sentences and the same assumption for the vector meaning of the subject $\ov{n}$. Assume further that for the vector of the object we have $\ov{np} = C'_1 \ov{n}_1 + C'_2 \ov{n}_2$. In this case, because the verb is an element of a rank 3 tensor space, that is $\ov{v} \in W \otimes S \otimes W$, it is not possible to express it as a matrix in two dimensions: indeed elements of  rank 3 tensor   spaces are cubes rather than matrices. But opting for not expanding the corresponding tensor and just denoting the cube of the verb by $\ov{v}$, the meaning vector of the transitive sentence with a quantified subject can be expressed as follows
\[
\left(
 \left ( \begin{array}{cc}
 C_1 & C_1\\ C_2 & C_2
  \end{array} \right)
 \ \odot \ 
\epsilon (\ov{v} \otimes\left (\begin{array}{c} C'_1 \\C'_2 \end{array} \right) ) \right)
 \quad \times \quad
 d\left ( \begin{array}{c} C_1 \\ C_2 \end{array} \right )
\]
The details of the tedious computations are as before, with the exception that in this case, first the verb has to be applied to its object. This is denoted by the application of the $\epsilon$ map to $\ov{v}$ and matrix form of $\ov{np}$. The rest of the computation is as before: one takes the point wise multiplication of this result with a form of diagonalisation of the vector of the subject, then applies this to the effect of the quantifier map $d$ on the vector of the subject. The result is a sentence vectors in $S$. 

