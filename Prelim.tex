
\section{Preliminaries}
\label{prelim}

\subsection{Generalised Quantifier Theory}

\subsection{Category Theoretic and Diagrammatic Definitions}
This subsection briefly reviews compact closed
categories and Frobenius algebras. For a formal presentation, see
\cite{KellyLaplaza80,Kock72}.  A compact closed category, $\cC$, has objects $A, B$; morphisms $f \colon A
\to B$; a monoidal tensor $A \otimes B$ that has a unit $I$; and for
each object $A$ two objects $A^r$ and $A^l$ together with the
following morphisms:
\begin{align*}
A \otimes A^r \stackrel{\epsilon_A^r} {\longrightarrow} \; &I
\stackrel{\eta_A^r}{\longrightarrow} A^r \otimes A \hspace{1cm}
A^l \otimes A \stackrel{\epsilon_A^l}{\longrightarrow} \; I
\stackrel{\eta_A^l}{\longrightarrow} A \otimes A^l\
\end{align*}
These morphisms satisfy the following equalities, sometimes
referred to as the \emph{yanking} equalities, where $1_A$ is the
identity morphism on object $A$:
\begin{align*}
& (1_A \otimes \epsilon_A^l) \circ (\eta_A^l \otimes 1_A)  = 1_A 
\hspace{1cm}
(\epsilon_A^r \otimes 1_A) \circ (1_A \otimes
  \eta_A^r)   = 1_A\\
& (\epsilon_A^l \otimes 1_A) \circ (1_{A^l} \otimes
  \eta_A^l) = 1_{A^l}  
    \hspace{1cm}
    (1_{A^r} \otimes \epsilon_A^r) \circ (\eta_A^r \otimes 1_{A^r}) = 1_{A^r}
\end{align*}
%
\noindent These express the fact the $A^l$ and $A^r$ are the left and right
adjoints, respectively, of $A$ in the 1-object bicategory whose
1-cells are objects of $\cC$.

%A pregroup \cite{Lambek08} is a partial order compact closed category,
%which we refer to as {\em Preg}. This means that the objects of {\em
%  Preg} are elements of a partially ordered monoid, and between any two
%objects $p,q \in \Preg$ there exists a morphism of type $p
%\to q$ iff $p \leq q$. Compositions of morphisms are obtained by
%transitivity and the identities by reflexivity of the partial
%order. The tensor of the category is the monoid multiplication, and
%the epsilon and eta maps are as follows:
%\begin{align*}
%\epsilon_p^r = p \cdot p^r \leq 1 &\hspace{1cm} \eta_p^r = 1 \leq p^r \cdot p \\
%\epsilon_p^l = p^l \cdot p \leq 1  &\hspace{1cm}  \eta_p^l = 1 \leq p \cdot p^l
%\end{align*}

A Frobenius algebra  in a   monoidal  category $({\cal
  C}, \otimes, I)$ is a tuple $(X,  \delta, \iota, \mu, \zeta)$ where,
for $X$ an object of ${\cal C}$, the triple $(X, \delta, \iota)$ is  an internal comonoid; 
i.e.~the following are  coassociative and counital  morphisms of ${\cal
  C}$:
\begin{align*}
\delta \colon X \to X \otimes X&\qquad& \iota \colon X \to I
\end{align*}
Moreover $(X, \mu, \zeta)$ is  an internal  monoid; i.e.~the following are  associative and unital  morphisms:
\begin{align*}
\mu \colon  X \otimes X \to X  &\qquad& \zeta \colon I \to X
\end{align*}
And finally the  $\delta$ and $\mu$ morphisms satisfy the
following \emph{Frobenius condition}:
\begin{align*}
\mbox{ $(\mu \otimes 1_X) \circ (1_X \otimes \delta) \ = \  \delta \circ \mu  \ = \  (1_X \otimes \mu) \circ (\delta \otimes 1_X)$}
\end{align*}
%MS
Informally, the  comultiplication $\delta$  dispatches the information contained in
one object into two objects, and the  multiplication $\mu$ unifies  the
information of two objects into one.




\medskip
\noindent
{\bf Finite Dimensional Vector Spaces.}
These structures  together with  linear maps  form a compact
closed category, which we refer to as $\FdVect$.  Finite dimensional
vector spaces $V, W$ are objects of this category; linear maps $f
\colon V \to W$ are its morphisms with composition being the
composition of linear maps. The tensor product $V
\otimes W$ is the 
linear algebraic tensor product,
%tensor of the category, 
whose unit is the scalar
field of vector spaces; in our case this is the field of reals
$\mathbb{R}$.  Here, there is  a naturual
isomorphism $V \otimes W \cong W \otimes V$. As a result of the
symmetry of the tensor, the two adjoints reduce to one and we obtain the  isomorphism $V^l \cong V^r \cong V^*$, 
where $V^*$ is the dual space of $V$. When the
basis vectors of the vector spaces are fixed, it is further the case
that we have $V^* \cong V$.

%Elements of vector spaces, i.e. vectors, are
%represented by morphisms from the unit of tensor to their corresponding
%vector space; that is $\overrightarrow{v} \in V$ is represented by the
%morphism $\mathbb{R} \stackrel{\overrightarrow{v}}{\longrightarrow}V$; by linearity this morphism is uniquely defined when setting $1\mapsto \overrightarrow{v}$.
%
Given a basis $\{r_i\}_i$ for a vector space $V$, the epsilon maps are
given by the inner product extended by linearity; i.e. we have:
\[
\epsilon^l  =  \epsilon^r \colon   V \otimes V \to \mathbb{R} \quad \mbox{given by} \quad
\sum_{ij} c_{ij} \ \psi_i \otimes \phi_j  \quad \mapsto \quad \sum_{ij} c_{ij} \langle \psi_i \mid \phi_j \rangle\]
Similarly, eta maps   are defined as follows:
\[
\eta^l = \eta^r \colon   \mathbb{R} \to V \otimes V
\quad \mbox{given by} \quad 
1 \; \mapsto \; \sum_i r_i \otimes r_i
\]
Any vector space $V$ with a fixed basis
$\{\ov{v_i}\}_i$ has a Frobenius algebra over it, explicitly given as follows, where $\delta_{ij}$ is the Kronecker delta.
%\begin{align*}
%\delta :: \ov{v_i} \mapsto \ov{v_i} \otimes \ov{v_i} &\qquad& \iota:: \ov{v_i} \mapsto 1\\
%\mu:: \ov{v_i} \otimes \ov{v_i} \mapsto \ov{v_i} &\qquad& \zeta:: 1 \mapsto \ov{v_i}  
%\end{align*}
\begin{eqnarray*}\label{eq:frob}
\delta  \colon V \to V \otimes V  & \quad \mbox{given by} \quad&  \ov{v_i} \mapsto \ov{v_i} \otimes \ov{v_i} \\
\mu \colon V \otimes V \to V  &\quad \mbox{given by} \quad & \ov{v_i} \otimes \ov{v_j} \mapsto 
\delta_{ij}\ov{v_i} \\
  \iota \colon V \to \mathbb{R} & \quad \mbox{given by} \quad&  \ov{v_i} \mapsto 1 \\
 \zeta \colon \mathbb{R} \to V  &\quad \mbox{given by} \quad& 1 \mapsto   \sum_i  \ov{v_i}  
\end{eqnarray*}


%Frobenius algebras over vector spaces with orthonormal bases are
%moreover \emph{isometric} and \emph{commutative}. But in benefit of space, we do not provide these conditions here. 
%A
%commutative Frobenius Algebra satisfies the following two conditions
%for $ \sigma : X \otimes Y \to Y \otimes X$, the symmetry morphism of
%$({\cal C}, \otimes, I)$:
%\[
%\sigma \circ \delta = \delta \hspace{2cm} \mu \circ \sigma = \mu
%\]
%An isometric Frobenius Algebra is one that satisfies the following axiom:
%\[
%\mu \circ \delta = 1
%\]
%The vector spaces of distributional models have
%fixed orthonormal bases; hence they have isometric commutative
%Frobenius algebras over them.
%An object $X$ with a Frobenius algebra over it is called
%\emph{classical} in \cite{CoeckePaquettePavlovic09}. 


%As an example, take $V$ to be a two dimensional space with
%basis $\{\ov{v_1}, \ov{v_2}\}$; then the basis of $V \otimes V$ is
%$\{\ov{v_1} \otimes \ov{v_1}, \ov{v_1} \otimes \ov{v_2}, \ov{v_2}
%\otimes \ov{v_1}, \ov{v_2} \otimes \ov{v_2}\}$. For a vector $v = a
%\ov{v_1} + b \ov{n_2}$ in $V$ we have:
%\begingroup
%\setlength{\arraycolsep}{2pt}
%\[
%\delta (v) = \delta \left(\begin{array}{c} a\\b\end{array}\right)
% = 
% \left(\begin{array}{cc} a&0\\0&b\end{array}\right) = a \, \ov{v_1} \otimes \ov{v_1} + b \, \ov{v_2} \otimes \ov{v_2} 
%\]
%And for a matrix $w = a \, \ov{v_1} \otimes \ov{v_1} + b \, \ov{v_1}
%\otimes \ov{v_2} + c \, \ov{v_2} \otimes \ov{v_1} + d \, \ov{v_2}
%\otimes \ov{v_2}$ in $V \otimes V$, we have:
%\[
%\mu (w) = \mu \left(\begin{array}{cc} a&b\\c&d\end{array}\right) = 
%\left(\begin{array}{c} a\\d\end{array}\right)
%= a \, \ov{v_1} + d \, \ov{v_2}
%\]
%\endgroup

\medskip
\noindent
{\bf Relations}.
Another important example of a  compact closed category is
$\Rel$, the cateogry of sets and relations. Here, $\otimes$ is
cartesian product with the singleton set as its unit $I = \{\star\}$, and $^*$ is identity on objects. Closure reduces to the
fact that a relation between sets $A\times B$ and $C$ is equivalently a relation between $A$ and $B \times C$.   Given a set $S$ with elements $s_i, s_j \in S$,  the epsilon and eta maps are given as follows:

\begin{eqnarray*}
&\epsilon^l  =  \epsilon^r &\colon   S \times S \to \{\star\} \quad \mbox{given by} \quad
\{((s_i, s_j), \star) \mid s_i, s_j \in S, s_i = s_j \}\\
&\eta^l = \eta^r& \colon   \{\star\}  \to S \times S
\quad \mbox{given by} \quad 
\{(\star, (s_i, s_j)) \mid s_i, s_j \in S, s_i = s_j\}
\end{eqnarray*}



Every object in $\Rel$  has a
Frobenius algebra over it given by the diagonal and codiagonal
relations, as described below: 



\begin{eqnarray*}
&\delta &\colon   S \to S \times S \quad \mbox{given by} \quad
\{(s_i, (s_j, s_k)) \mid s_i, s_j, s_k \in S, s_i = s_j  = s_k\}\\
& \mu & \colon   S \times S \to S
\quad \mbox{given by} \quad 
\{(s_i, s_j), s_k) \mid s_i, s_j, s_k\in S, s_i = s_j= s_k\}\\
& \iota& \colon S \to  \{\star\}    \quad \mbox{given by} \quad \{(s_i, \star) \mid s_i \in S\}\\
&\zeta& \colon  \{\star\}  \to S  \quad \mbox{given by} \quad \{(\star, s_i) \mid s_i \in S\}
\end{eqnarray*}

For the details of verifying that for each of the two examples above,  the corresponding conditions hold see \cite{CoeckePaq}. 

\section{String Diagrams} 
\label{string}

The framework of compact closed categories and Frobenius algebras
comes with a complete diagrammatic calculus that visualises
derivations, and which also simplifies the
categorical and vector space computations. Morphisms are depicted by
boxes and objects by lines, representing their identity morphisms. For
instance a morphism $f \colon A \to B$, and an object $A$ with the
identity arrow $1_A \colon A \to A$, are depicted as follows:

\begin{center}
  \tikzfig{compact-diag}
\end{center}

The tensor products of the objects and morphisms are depicted by
juxtaposing their diagrams side by side, whereas compositions of
morphisms are depicted by putting one on top of the other; for instance
the object $A \otimes B$, and the morphisms $f \otimes g$ and $f \circ
h$, for $f \colon A \to B, g \colon C \to D$, and $h \colon B \to C$,
are depicted as follows:

\begin{center}
  \tikzfig{compact-diag-tensor}
\end{center}

The $\epsilon$ maps are depicted by cups, $\eta$ maps by caps, and
yanking by their composition and straightening of the strings.  For
instance, the diagrams for $\epsilon^l \colon A^l \otimes A \to I$,
$\eta \colon I \to A\otimes A^l$ and $(\epsilon^l \otimes 1_A) \circ
(1_A \otimes \eta^l) = 1_A$ are as follows:

\begin{center}
  \tikzfig{compact-cap-cup}
  \qquad
    \tikzfig{compact-yank}
\end{center}

 
As for Frobenius algebras, the diagrams for the  monoid and  comonoid 
morphisms are as follows:

\begin{center}
\tikzfig{comp-alg-coalg}
\end{center} 
 
\noindent
with the Frobenius condition being depicted as:

\begin{center}
\tikzfig{equation}
\end{center} 

%\noindent The commutativity conditions are depicted as follows:
%
%\begin{center}
%\tikzfig{comm}
%\end{center} 
%
%\noindent The isometry condition is depicted as follows:
%
%\begin{center}
%\tikzfig{isom}
%\end{center} 
%

\noindent
The defining axioms   guarantee that any picture depicting a
Frobenius computation can be reduced to a normal form that only
depends on the number of input and output strings of the nodes,
independent of the topology. 
These normal forms can be simplified to so-called `spiders': 
%This property, known as the \emph{spider  equality}, is depicted as follows:

\[
\tikzfig{spider}
\]

In the category $\FdVect$, apart from spaces $V,W$, which are
objects of the category, we also have vectors $\ov{v}, \ov{w}$. These
are depicted by their representing morphisms and as triangles with a
number of strings emanating from them. The number of strings of a
triangle denote the tensor rank of the vector; for instance, the
diagrams for $\ov{v} \in V, \ov{v'} \in V \otimes W$, and $\ov{v''}
\in V \otimes W \otimes Z$ are as follows:

\begin{center}
  \tikzfig{compact-diag-triangle}  
\end{center} 


%%% Local Variables: 
%%% mode: latex
%%% TeX-master: "Quant"
%%% End: 
