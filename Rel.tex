\section{Relational Interpretation}
%%% Local Variables: 
%%% mode: latex
%%% TeX-master: "Quant"
%%% End: 


The category $\Rel$ of sets and relations is a basic example of a
symmetric monoidal compact closed category. The tensor is Cartesian
product, $\times$, with $1$, the one element set, the unit; $*$ is
identity. Monoidal closure is therefore just the product. This
formally exhibits the correspondence between relations and subsets of
the Cartesian product, because a relation $ 1 \to X \x Y $ is clearly
just a subset of $X \x Y$, and by the monoidal closure isomorphism it
is also a relation $ X \to Y $~; and of course also a relation $X \x Y
\to 1$ by converse.

\ondrej{This is probably best put where compact
  closed cats are introduced.}

The frobenius maps $\delta$ is just 
the diagonal relation $x \sim (x,x)$ and $\mu$ its transpose; $\eta : 1
\to X$ is the so-called \emph{fan} relation: $\ast \sim x, ~  \forall
x$, $\iota$ its converse. The required axioms are easilly checked. 
\ondrej{This is probably best put where Frobenius structur is introduced.}


In order to introduce quantification in $\Rel$, we first discuss the
well know case in $\Set$ where the existential and universal
quantifiers are left and right adjoints, respectively, to inverse
image functor.
%
For a set $X$, the set of functions from $X$ to the two-element set,
$2^X$, is demonstrably the powerset of $X$.  For a function $f : X \to
Y$, precomposition with $f$ is internally the map $2^f : 2^Y \to
2^X$. It takes a $h$ to $hf$.  It has both left and right adjoints
with respect to the pointwise subset ordering. In more detail, the
powerset of $X$ ordered by subset inclusion is isomorphic to the set
of characteristic functions on $X$, $2^X$, ordered pointwise by the
ordering $0 < 1$.  Any pre-ordered set can be seen as a category. The
function $2^f$ becomes a functor from the category $2^Y$ to
$2^X$. This functor has both adjoints. It is well known that when $f$
is the first projection $\pi_1 : X\x Y \to X$, $2^{\pi_1}$ corresponds
to \emph{weakening} in the sense that subsets of $X$ are predicates on
$X$, and $2^{\pi_1}$ takes predicates on $X$ to predicates on $X\x Y$
where the element of the second component $Y$ doesn't occur.  The left
adjoint to $2^{\pi_1}$ now interprets existential quantification in first
order predicate logic, and the right adjoint interprets universal
quantification. In detail, 
the left adjoint, $\exists_f : 2^X \to 2^Y$, is defined by 
\[\exists_f (X' \subseteq X) = \{ y \in Y ~|~ \exists x \in X. f x = y
\wedge x \in X'\} \]
i.e. the image of $X'$ under $f$. It is also known as the
\emph{direct image} of $f$.
The right adjoint $\forall_f : 2^X
\to 2^Y$ is defined by 
%
\[
\forall_f (X' \subseteq X) = \{ y \in Y ~|~ \forall x \in X. f x = y
\implies x \in X' \} \] 
%


Note that there may be many other functions from $2^{X\x Y}$ to $2^X$,
such as the one sending an $f$ to the $g$ for which
$g(x) = 1$ if and only if $f(x,y) = 1$ for exactly two distinct values
of $y$. This quantifier could be called ``two''. Another example is
``none'' for which $g(x)=1$ iff $f(x,y) = 0$ always. This exhibits the 
existential and universal quantifiers as two extreme cases of a
spectrum of all possible quantifiers.  

In the following we assume for simplicity that the ambient context,
$X$, is empty, i.e. $X = 1$, and then quantifiers become certain
functions $2^Y \to 2$. Everything we say generalises
straightforwardly.  Note that in this case, the existential quantifier
becomes the characteristic function of nonempty subsets of $Y$; the
universal quantifier is the characteristic function of the singleton
$\{Y\}$, and ``two'' is the characteristic function of two
element sets.

In single sorted first order predicate logic, quantification ranges
over the whole domain $Y$, which is simply assumed to be fixed, and
often implicit, throught the formula. For example, in $\exists x. x >
0$, the bound variable, $x$,
ranges over some ordered set. In linguistics, however, the range of
quantification is explicitly stated, as in ``all men sleep'', where
the category 
``men'' restricts the range of the quantifier, ``all'', to just
men. In other words, the quantifier is ``all men'', and ``all'' is a
particle which expects a category to become a quantifier. Formally,
when we fix a domain of all possible subjects of quantification, $X$,
a quantifier $Q$ must have type 
\begin{equation}\label{eq:q}
2^X \to 2^{2^X}
\end{equation}

Here, the first argument to $Q$ is a subset of the universal domain,
the range of quantification. The result is a quantifier which
possibly ignores everything that is outside the range. 

For instance, ``some'' ($\exists$) takes a subset
$\;\mathrm{men}\subseteq X\;$ to all nonempty subsets of men; all
($\forall$) maps $\;\mathrm{men}\subseteq X\;$ to $ \{ \mathrm{men}
\}$; ``two'' takes $~\mathrm{men} \subseteq X$~ to two-element sets of
men.

Now note, that the $\Rel$ is equivalent to the Kleisli category of the
powerset monad $2^-$, i.e. an arrow of type \eqref{eq:q} is precisely
a relation $2^X \to 2^X$. 

\renewcommand{\wp}{2^}
It follows that in order to interpret quantification in relations we
must consider relations over powersets. 
To this end, we lift $\wp-$ to $\Rel$. Define $\wp- : \Rel \to
\Rel$ on objects $X \mapsto \wp X$, and on arrows as follows: a
relations $R : X \to Y$ is mapped to that relation $\wp R : \wp X \to
\wp Y$ which relates every subset of $X$ to all subsets of $Y$ which
arise from $X$ by replacement of elements under $R$. In other words,
when relations are understood as nondeterministic partial functions
$\wp R$ captures all possible nondeterminism on collections of
elements. More formally, identify $\Rel$ with the Kleisli
category of the covariant powerset functor whose action on functions
is the direct image. There is a distributive law, a natural
transformation $\lambda_X : \wp{\wp X} \to \wp{\wp X}$, which takes a
set of subsets of $X$ into the set of its crossections. 

For example: 
\begin{align*}
\lambda (\{\{a,b\},\{c\}\}) & \quad = \quad \{\{a,c\},\{b,c\}\}\\
\lambda (\{\{a,b\},\{c\},\emptyset\}) &\quad = \quad \emptyset
\end{align*}

\begin{quote}
{\bf O.R.} check it respects the monad structure, at least the necessary
  half of it 
\end{quote}

It follows that $\wp-$ lifts to $\Rel$ where a
relation $R :  X \to \wp Y$ is taken to $\wp R : \wp X \xrightarrow{R} \wp {\wp Y}
\xrightarrow{\lambda_Y} \wp {\wp Y}$.


Example: lifted frobenius

\begin{quote}
{\bf O.R.} Now it all works out :)
\end{quote}


Examples of calculations with determiners: exists, all, some, two





%For this part, we work in the category $Rel$ of sets and relations. This is compact closed as follows \cite{BobEric}. 
%
%Take $N$ to be a vector space with a fixed basis ${\cal B}$, where ${\cal B}$ is a finite set. A  basis vector of $N$  is denoted by $\ov{n}_i$. Given a set of individuals ${\cal U}$,  each  individual is mapped to a basis vector of $N$, the map is denoted by $\pi \colon {\cal U} \to {\cal B}$.  Thus a subset of basis vectors of $N$ represents a subset of individuals.  For example,    the sum $\sum_i \ov{n}_i$ denotes the `men' subset of individuals, where $i$ ranges over the basis vectors that are mappings of the individuals that are men. 
%
%Take $S$ to be the one dimensional space free over the singleton $\ov{1}$. The  zero vector represents false, and any nonzero value represents a degree of truth. 
%
%A transitive verb $w$, which is a vector in the space $N \otimes S
%\otimes N$, is represented by 
%\[
%\ov{w} := \sum_{ij} \ov{\{{n}\}}_i \otimes \ov{1}  \otimes \ov{\{n\}}_j, \quad 
%\text{if} \  \pi^{-1}({n}_i) \ w\mbox{'s} \  \pi^{-1}({n}_j)
%\]
% For example, the verb ``stroke'', denoted by $\ov{stroke}$, is represented by $
%\sum_{(i,j) \in  R_{stroke}} \ov{\{n\}}_i \otimes  \ov{1}
%\otimes \ov{\{n\}}_j$, for  $R_{stroke}$  the set of all pairs $(i,j)$ such that $\pi^{-1}(\ov{n}_i)$ strokes $\pi^{-1}(\ov{n}_j)$.  An intransitive verb ``sneeze'', denoted by $\ov{sneeze}$, is represented by $\sum_{i \in R_{sleep}} \ov{\{n\}}_i \otimes \ov{1}$, such that $\pi^{-1}(\ov{n}_i)$ sneezes.  
%
%
%The meaning of the sentence ``Det Sbj Verb''is a vector obtained by computing the following, which corresponds to the categorical morphism of the normalised diagram of the sentence.  
%
%\[
%(\epsilon_N \otimes 1_S) \circ (Det \otimes  \mu_N \otimes 1_S) \circ (\delta_N \otimes 1_{N \otimes S})\Big(\ov{Sbj} \otimes \ov{Verb}\Big)
%\]
%This is computed in three steps. First we compute the following
%
%\begin{align*}
%(\delta_N \otimes 1_{N \otimes S})\Big(\ov{Sbj} \otimes \ov{Verb}\Big) = 
%(\delta_N \otimes 1_{N \otimes S})\Big((\sum_i \ov{n}_i) \otimes (\sum_j \ov{n}_j  \otimes \ov{1}) \Big) =\\
%\delta_N(\sum_i \ov{n}_i) \otimes (\sum_j \ov{n}_j \otimes \ov{1}) = (\sum_i \ov{n}_i \otimes \ov{n}_i) \otimes (\sum_j \ov{n}_j \otimes \ov{1})
%\end{align*}
%
%\noindent
%Then we proceed by
%
%\begin{align*}
%(Det \otimes  \mu_N \otimes 1_S)\Big(\sum_i \ov{n}_i \otimes \ov{n}_i) \otimes (\sum_j \ov{n}_j \otimes \ov{1}\Big) = \\
%Det(\sum_i \ov{n}_i) \otimes  \mu_N(\sum_i \sum_j \ov{n}_i \otimes \ov{n}_j) \otimes 1_S(\ov{1}) = \\
%Det(\sum_i \ov{n}_i) \otimes  (\sum_i \sigma_{ij} \ov{n}_i) \otimes \ov{1} 
%\end{align*}
%
%\noindent
%The final step is as follows:
%
%\begin{align*}
%(\epsilon_{N} \otimes 1_S)  \Big(Det(\sum_i \ov{n}_i) \otimes (\sum_i \sigma_{ij} \ov{n}_i) \otimes \ov{1} \Big) =   \langle \ov{w}_k \mid \ov{\{\sum_i \ov{n}_i\}} \rangle  \otimes \ov{1}
%\end{align*}
%
%\noindent 
%where $Det (\sum_i \ov{n}_i) = \sum_k \ov{w}_k$, for $w_k \subseteq B_N$ where $B_N$ is the set of basis vectors of $N$.  Meanings of sentences with quantified objects and transitive verbs are computed in an identical fashion. 
%
%
%
%
%As an example suppose we have  a set of men $\{m_1, m_2\}$ and a set of cats $\{c_1, c_2\}$. Now suppose that one of the men $m_1$ and all of the cats  sleep, that is we have
%\[
%\ov{sleep} \ = \  \ov{m}_1 \otimes \ov{1} + \ov{c}_1 \otimes \ov{1} + \ov{c}_2 \otimes \ov{1}
%\]
%Here are some sample sentences:
%\begin{enumerate}
%\item The meaning  of the sentence `some men sleep', is computed in three steps. In the first step, we obtain
%\[
%(\ov{m}_1 \otimes \ov{m}_1 + \ov{m}_2 \otimes \ov{m}_2)  \otimes ( \ov{m}_1 \otimes \ov{1} + \ov{c}_1 \otimes \ov{1} + \ov{c}_2 \otimes \ov{1})
%\]
%In the second step we obtain
%\[
%Some(\ov{m}_1 + \ov{m}_2) \otimes {\cal P}(\ov{m}_1) \otimes \ov{1} = 
%(\ov{\{\ov{m}_1\}} + \ov{\{\ov{m}_2\}} + \ov{\{\ov{m}_1, \ov{m}_2\}}) \otimes (\ov{\{\ov{m}_1\}}) \otimes \ov{1}
% \]
%In the last step we obtain $\ov{1}$, since we have $\left \langle \ov{\{\ov{m}_1\}} \mid \ov{\{\ov{m}_1\}}) \right \rangle = 1$. This  means that the meaning of this sentence is true.  
%\item The meaning of `all men sleep' would be false. Since, in this case, in the second step we would obtain
%\[
%All(\ov{m}_1 + \ov{m}_2) \otimes {\cal P}(\ov{m}_1) \otimes \ov{1} = 
%(\ov{\{\ov{m}_1, \ov{m}_2\}}) \otimes (\ov{\{\ov{m}_1\}}) \otimes \ov{1}
% \]
%which would result to a 0 in the third step, since $\left \langle \ov{\{\ov{m}_1, \ov{m}_2\}} \mid \ov{\{\ov{m}_1\}}) \right \rangle = 0$.  
%\item The sentence `half of  men sleep' would also be true, since in the second step we obtain
%\[
%Half(\ov{m}_1 + \ov{m}_2) \otimes {\cal P}(\ov{m}_1) \otimes \ov{1} = 
%(\ov{\{\ov{m}_1\}} + \ov{\{\ov{m}_2\}}) \otimes (\ov{\{\ov{m}_1\}}) \otimes \ov{1}
% \]
%since we  have $\left \langle \ov{\{\ov{m}_1\}} \mid \ov{\{\ov{m}_1\}}) \right \rangle = 1$, in the third step we will obtain a $\ov{1}$.  
%\item The sentence `no man sleeps' will be false, since in the second step we obtain
%\[
%No(\ov{m}_1 + \ov{m}_2) \otimes {\cal P}(\ov{m}_1) \otimes \ov{1} = 
%(\ov{\{ \ \}}) \otimes (\ov{\{\ov{m}_1\}}) \otimes \ov{1}
% \]
% This, in the third step,  will results in $\left \langle \ov{\{\ \}} \mid \ov{\{\ov{m}_1\}}) \right \rangle = 0$.
%\end{enumerate}


%%% Local Variables: 
%%% mode: latex
%%% TeX-master: "Quant"
%%% End: 
