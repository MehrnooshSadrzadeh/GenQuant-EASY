\section{Relational Interpretation}
%%% Local Variables: 
%%% mode: latex
%%% TeX-master: "Quant"
%%% End: 


Relational interpretation of Frobenius and the compact structure. Also
point out the isomorphisms $1 \to X\x Y // X \to Y // X \x Y \to 1$.

Intended interpretation of $\Det$ and the need for matching powerset
levels as a motivation for Finite Powerset lifting. 

\medskip
For a function $f : X \to Y$, the map $2^f : 2^Y \to 2^X$ has both a
left and right adjoints with respect to the pointwise subset
ordering. In more detail, the powerset of $X$, $2^X$, ordered by
subset inclusion  can be seen as a
category and $2^f$ as a functor from the category $2^Y$ to $2^X$. This
functor has both adjoints. It is well known that when $f$ is the first
projection $\pi_1 : X\x Y \to X$, the left adjoint interprets
existential quantification in first order predicate logic, and the
right adjoint interprets universal quantification. 

The definitions of the adjoints are in detail as follows. 
The left adjoint $\exists_f : 2^X \to 2^Y$, also called the
\emph{direct image} of $f$, is defined by 
\[\exists_f (X' \subseteq X) = \{ y \in Y ~|~ \exists x \in X. f x = y
\wedge x \in X'\} \]
i.e. the image of $X'$ under $f$. The right adjoint $\forall_f : 2^X
\to 2^Y$ is defined by 
%
\[
\forall_f (X' \subseteq X) = \{ y \in Y ~|~ \forall x \in X. f x = y
\implies x \in X' \} \] 
%


Not that there may be many other functions from $2^{X\x Y}$ to $2^X$,
such as the one sending an $f$ to the $g$ for which
$g(x) = 1$ if and only if $f(x,y) = 1$ for exactly two distinct values
of $y$. This quantifier we could call ``two''; or
``none'' for which $g(x)=1$ iff $f(x,y) = 0$ always. Thus the
existential and universal quantifier are the two extreme cases. 

Unfortunately, $\Rel$ is not Cartesian closed with respect to the
product. More precisely, $\Rel$ is monoidal closed with respect to the
product of sets, $\times$,  but $\times$ is not the
Cartesian product in $\Rel$. So we cannot replicate the $\Set$
case directly. 

We must lift $\wp$ to $\Rel$. To this end, define $\wp : \Rel \to
\Rel$ on objects $X \mapsto \wp X$, and on arrows as follows: a
relations $R : X \to Y$ is mapped to that relation $\wp R : \wp X \to
\wp Y$ which relates every subset of $X$ to all subsets of $Y$ which
arise from $X$ by replacement of element under $R$. In other words,
when relations are understood as nondeterministic partial functions
$\wp R$ captures all possible nondeterminism on collections of
elements. More formally, identify $\Rel$ with $\Set_\wp$, the Kleisli
category of the covariant powerset functor whose action on functions
is the direct image. There is a distributive law, a natural
transformation $\lambda_X : \wp \wp X \to \wp \wp X$, which takes a
set of subsets of $X$ into the set of its crossections. 

For example: 
\begin{align*}
\lambda (\{\{a,b\},\{c\}\}) & \quad = \quad \{\{a,c\},\{b,c\}\}\\
\lambda (\{\{a,b\},\{c\},\emptyset\}) &\quad = \quad \emptyset
\end{align*}

\begin{quote}
{\bf O.R.} check it respects the monad structure, at least the necessary
  half of it 
\end{quote}

It follows that $\wp$ lifts to $\Rel$, equivalently the Kleisli category of $\wp$, where a
relation $R :  X \to \wp Y$ is taken to $\wp R : \wp X \xrightarrow{R} \wp \wp Y
\xrightarrow{\lambda_Y} \wp \wp Y$.


Example: lifted frobenius

\begin{quote}
{\bf O.R.} Now it seems I have missed the point, because generalised
quantifiers work in a dirrent way, it seems. Based on
cardinality. Let's see how these two relate... trouble is I haven't
slept well and can't really think :-/ I'll come to this a bit later. 
\end{quote}


Examples of determiners with some calculations: exists, all, some, two





%For this part, we work in the category $Rel$ of sets and relations. This is compact closed as follows \cite{BobEric}. 
%
%Take $N$ to be a vector space with a fixed basis ${\cal B}$, where ${\cal B}$ is a finite set. A  basis vector of $N$  is denoted by $\ov{n}_i$. Given a set of individuals ${\cal U}$,  each  individual is mapped to a basis vector of $N$, the map is denoted by $\pi \colon {\cal U} \to {\cal B}$.  Thus a subset of basis vectors of $N$ represents a subset of individuals.  For example,    the sum $\sum_i \ov{n}_i$ denotes the `men' subset of individuals, where $i$ ranges over the basis vectors that are mappings of the individuals that are men. 
%
%Take $S$ to be the one dimensional space free over the singleton $\ov{1}$. The  zero vector represents false, and any nonzero value represents a degree of truth. 
%
%A transitive verb $w$, which is a vector in the space $N \otimes S
%\otimes N$, is represented by 
%\[
%\ov{w} := \sum_{ij} \ov{\{{n}\}}_i \otimes \ov{1}  \otimes \ov{\{n\}}_j, \quad 
%\text{if} \  \pi^{-1}({n}_i) \ w\mbox{'s} \  \pi^{-1}({n}_j)
%\]
% For example, the verb ``stroke'', denoted by $\ov{stroke}$, is represented by $
%\sum_{(i,j) \in  R_{stroke}} \ov{\{n\}}_i \otimes  \ov{1}
%\otimes \ov{\{n\}}_j$, for  $R_{stroke}$  the set of all pairs $(i,j)$ such that $\pi^{-1}(\ov{n}_i)$ strokes $\pi^{-1}(\ov{n}_j)$.  An intransitive verb ``sneeze'', denoted by $\ov{sneeze}$, is represented by $\sum_{i \in R_{sleep}} \ov{\{n\}}_i \otimes \ov{1}$, such that $\pi^{-1}(\ov{n}_i)$ sneezes.  
%
%
%The meaning of the sentence ``Det Sbj Verb''is a vector obtained by computing the following, which corresponds to the categorical morphism of the normalised diagram of the sentence.  
%
%\[
%(\epsilon_N \otimes 1_S) \circ (Det \otimes  \mu_N \otimes 1_S) \circ (\delta_N \otimes 1_{N \otimes S})\Big(\ov{Sbj} \otimes \ov{Verb}\Big)
%\]
%This is computed in three steps. First we compute the following
%
%\begin{align*}
%(\delta_N \otimes 1_{N \otimes S})\Big(\ov{Sbj} \otimes \ov{Verb}\Big) = 
%(\delta_N \otimes 1_{N \otimes S})\Big((\sum_i \ov{n}_i) \otimes (\sum_j \ov{n}_j  \otimes \ov{1}) \Big) =\\
%\delta_N(\sum_i \ov{n}_i) \otimes (\sum_j \ov{n}_j \otimes \ov{1}) = (\sum_i \ov{n}_i \otimes \ov{n}_i) \otimes (\sum_j \ov{n}_j \otimes \ov{1})
%\end{align*}
%
%\noindent
%Then we proceed by
%
%\begin{align*}
%(Det \otimes  \mu_N \otimes 1_S)\Big(\sum_i \ov{n}_i \otimes \ov{n}_i) \otimes (\sum_j \ov{n}_j \otimes \ov{1}\Big) = \\
%Det(\sum_i \ov{n}_i) \otimes  \mu_N(\sum_i \sum_j \ov{n}_i \otimes \ov{n}_j) \otimes 1_S(\ov{1}) = \\
%Det(\sum_i \ov{n}_i) \otimes  (\sum_i \sigma_{ij} \ov{n}_i) \otimes \ov{1} 
%\end{align*}
%
%\noindent
%The final step is as follows:
%
%\begin{align*}
%(\epsilon_{N} \otimes 1_S)  \Big(Det(\sum_i \ov{n}_i) \otimes (\sum_i \sigma_{ij} \ov{n}_i) \otimes \ov{1} \Big) =   \langle \ov{w}_k \mid \ov{\{\sum_i \ov{n}_i\}} \rangle  \otimes \ov{1}
%\end{align*}
%
%\noindent 
%where $Det (\sum_i \ov{n}_i) = \sum_k \ov{w}_k$, for $w_k \subseteq B_N$ where $B_N$ is the set of basis vectors of $N$.  Meanings of sentences with quantified objects and transitive verbs are computed in an identical fashion. 
%
%
%
%
%As an example suppose we have  a set of men $\{m_1, m_2\}$ and a set of cats $\{c_1, c_2\}$. Now suppose that one of the men $m_1$ and all of the cats  sleep, that is we have
%\[
%\ov{sleep} \ = \  \ov{m}_1 \otimes \ov{1} + \ov{c}_1 \otimes \ov{1} + \ov{c}_2 \otimes \ov{1}
%\]
%Here are some sample sentences:
%\begin{enumerate}
%\item The meaning  of the sentence `some men sleep', is computed in three steps. In the first step, we obtain
%\[
%(\ov{m}_1 \otimes \ov{m}_1 + \ov{m}_2 \otimes \ov{m}_2)  \otimes ( \ov{m}_1 \otimes \ov{1} + \ov{c}_1 \otimes \ov{1} + \ov{c}_2 \otimes \ov{1})
%\]
%In the second step we obtain
%\[
%Some(\ov{m}_1 + \ov{m}_2) \otimes {\cal P}(\ov{m}_1) \otimes \ov{1} = 
%(\ov{\{\ov{m}_1\}} + \ov{\{\ov{m}_2\}} + \ov{\{\ov{m}_1, \ov{m}_2\}}) \otimes (\ov{\{\ov{m}_1\}}) \otimes \ov{1}
% \]
%In the last step we obtain $\ov{1}$, since we have $\left \langle \ov{\{\ov{m}_1\}} \mid \ov{\{\ov{m}_1\}}) \right \rangle = 1$. This  means that the meaning of this sentence is true.  
%\item The meaning of `all men sleep' would be false. Since, in this case, in the second step we would obtain
%\[
%All(\ov{m}_1 + \ov{m}_2) \otimes {\cal P}(\ov{m}_1) \otimes \ov{1} = 
%(\ov{\{\ov{m}_1, \ov{m}_2\}}) \otimes (\ov{\{\ov{m}_1\}}) \otimes \ov{1}
% \]
%which would result to a 0 in the third step, since $\left \langle \ov{\{\ov{m}_1, \ov{m}_2\}} \mid \ov{\{\ov{m}_1\}}) \right \rangle = 0$.  
%\item The sentence `half of  men sleep' would also be true, since in the second step we obtain
%\[
%Half(\ov{m}_1 + \ov{m}_2) \otimes {\cal P}(\ov{m}_1) \otimes \ov{1} = 
%(\ov{\{\ov{m}_1\}} + \ov{\{\ov{m}_2\}}) \otimes (\ov{\{\ov{m}_1\}}) \otimes \ov{1}
% \]
%since we  have $\left \langle \ov{\{\ov{m}_1\}} \mid \ov{\{\ov{m}_1\}}) \right \rangle = 1$, in the third step we will obtain a $\ov{1}$.  
%\item The sentence `no man sleeps' will be false, since in the second step we obtain
%\[
%No(\ov{m}_1 + \ov{m}_2) \otimes {\cal P}(\ov{m}_1) \otimes \ov{1} = 
%(\ov{\{ \ \}}) \otimes (\ov{\{\ov{m}_1\}}) \otimes \ov{1}
% \]
% This, in the third step,  will results in $\left \langle \ov{\{\ \}} \mid \ov{\{\ov{m}_1\}}) \right \rangle = 0$.
%\end{enumerate}


%%% Local Variables: 
%%% mode: latex
%%% TeX-master: "Quant"
%%% End: 
