\newcommand{\FinSet}{\mathrm{FinSet}}
\section{Compact Closed Semantics}
%
Categorical logic is an algebra of subobjects. For instance, a
predicate on a set corresponds to its subset, intersection to
conjunction, etc. Likewise, one would like to identify a predicate on
a vector space with a subspace. However, $\Set$ and $\FdVect$ are very
different categories so one cannot expect the logic of sets to work
for vector spaces without some strong additional assumptions. A
stepping stone on the route from sets to vector spaces, which is
important in its own right, is the category of sets and relations,
$\Rel$. $\Rel$ is somewhat like $\FdVect$ in that it is also dagger
compact closed, and the empty set, $\emptyset$, is both initial and
terminal. 
% On the other hand, there are many differences: $\FdVect$ is
% abelian; it has pullbacks and bi-products $\oplus$ but $A \oplus 0
% \cong A$ in $\FdVect$ whereas $A \times 0 \cong 0$ in $\Set$ and
% $\Rel$.

Here we start to investigate how far one can get in $\FdVect$ by
working in analogy with $\Rel$.  First and foremost, the compact tensor of $\Rel$ is the cartesian product. Its analogical version is the compact tensor of $FVect$ is the tensor product of  vector spaces.  Further, the  unit of tensor in $\Rel$ is the singleton set $\{\star\}$; its analogical version is the unit of tensor in $\FdVect$, that  is the scalar field $\mathbb{K}$. In the former case, we have two relations on $\{\star\}$: the identity and the empty set. In the latter, each linear map from $\mathbb{K}$ to itself represents a scalar. Both of these units form monoids, that of the former case is the monoid of Booleans and that of the latter case is the scalar monoid. They both are also rings, we have the scalar addition in the former case, which corresponds to the Boolean disjunction in the latter case. 


Hence, by moving from $\Rel$ to $\FdVect$, we are replacing propositional
truth values $\mathsf{t}$ and $\mathsf{f}$ by positive real numbers
$1$ and $0$, and the propositional connectives $\wedge$ and $\vee$ by
$\times$ and $+$, respectively.  I.e. the property of being related
becomes a number, a \emph{degree of relatedness}.  Formally, we
proceede by formulating set-theoretical logic in terms of the compact
structure in $\Rel$. Then by replacing the compact structure of $\Rel$
by the one in $\FdVect$ we obtain an analogous but very
different notion of \emph{generalised} subset. {\tt what is a generalised subset?}

An important assumption we make about our vector spaces is that we
assume they come with a fixed basis, which is equivalent to having a
Frobenius algebra $(\mu_A,\eta_A,\delta_A,\iota_A)$ over each vector
space. A vector space with Frobenius algebra over it is called
a \emph{classical structure} in \cite{CoeckePaquettePavlovic}. Classical
structures share many properties with sets. However, even though one
can define an appropriate notion of morphism for classical structures
so that they form a category, $\FdVect_c$, unless 
composition is redefined in a radical way there is not even a functor
from $\Rel$ to $\FdVect_c$.
{\tt ... may be explain why this the case briefly?}

