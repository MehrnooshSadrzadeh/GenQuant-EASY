\section{Abstract Compact Closed Semantics}

%%% Local Variables: 
%%% mode: latex
%%% TeX-master: "Quant"
%%% End: 

An abstract compact closed categorical model for the language  ${\cal L}_{\Sigma} = ({\cal X}_S, {\cal R})$  is  a tuple $({\cal C}_{W,S}, \overline{\semantics{\ }})$ where  ${\cal C}$  is a compact closed category  with two distinguished objects $W$ and $S$ where  $W$ has a Frobenius algebra on it and $\overline{\semantics{\ }}$ is a function that   assigns  morphisms from this category to  expression of the language. The interpretations of  the basic expressions  are as follows:

\begin{eqnarray*}
(np, \text{NP}) \in {\cal X}_S   &\quad \implies \quad& \overline{\semantics{np}} \colon I \to W \\
(n, \text{N}) \in {\cal X}_S   &\quad \implies \quad& \overline{\semantics{n}} \colon I \to W \\
(vp,  \text{VP}) \in {\cal X}_S &\quad \implies \quad& \overline{\semantics{vp}}   \colon I \to W^r \otimes S \\
(v, \text{V}) \in {\cal X}_S &\quad \implies \quad& \overline{\semantics{v}}  \colon I \to W^r \otimes S \otimes W^l  \\
(d, \text{Det}) \in  {\cal X}_S & \quad \implies \quad & \overline{\semantics{d}} \colon W \to W
\end{eqnarray*}

\noindent
The diagrammatic semantics of the above interpretations are as follows:

\begin{center}
\tikzfig{N-NP} \qquad \tikzfig{V-VP} \qquad \tikzfig{Det}
\end{center}

\noindent
Noun phrases and nouns are elements  within the  object $W$; the former is a singleton element and the latter not necessarily so.  The abstract language and its diagrammatic representation do not have means of distinguishing the two;  when we instantiate these to concrete categories the difference between them becomes evident. Verb phrases are elements within the object $W^r \otimes S$; the intuition behind this representation is that in a compact closed category we have that $W^r \otimes S \cong W \to S$, where $W^r \to S = Hom(W,S)$ is an internal hom object of the category, coming from its monoidal closeness.  Hence,  we are modelling verb phrases as morphisms  with  input $W$ and  output $S$. Similarly, verbs are elements within the object $W^r \otimes S \otimes W^r$, equivalent to morphisms $W \otimes W \to S$ with pairs of  input  from $W$ and output  $S$. 


\bigskip
The  interpretations of the expressions  generated by the rules are defined  by induction as follows:




\begin{itemize}
\item  $\overline{\semantics{\mbox{Det N}}} := \overline{\semantics{d}} \circ \overline{\semantics{n}}$, where  $
\overline{\semantics{d}} \circ \overline{\semantics{n}} \ \cong \ (\epsilon_W \otimes 1_W) \circ (\overline{\semantics{d}} \otimes \mu_W) \circ \sigma_W \circ \overline{\semantics{n}}$. Diagrammatically, we have:

\begin{center}
\tikzfig{Det-N-simple} \qquad $\cong$ \qquad \tikzfig{Det-N}  
\end{center}

\noindent
This condition expresses the `living on\ property.  Intuitively,  by using the axioms of compact closed categories and Frobenius algebras,  the right hand side diagram above simplifies to the following left hand side diagram below, which in turn is equivalent to the right hand side diagram below:

\begin{center}
 \tikzfig{Det-N-norm} \qquad $\cong$ \qquad  \tikzfig{Det-N-norm2} 
 \end{center}
 
 \noindent
 According to the  diagram on the right hand side, semantics of  $\overline{\semantics{d}} \circ \overline{\semantics{n}}$ is an element of $W$ which is equivalent to the element obtained by making a copy (via the Frobenius map $\delta_W$) of the noun  in $W$,  applying the determiner  to one copy and taking the intersection of the other copy (via the Frobenius map $\mu_W$) with $W$. 
 

\item $\overline{\semantics{\mbox{V NP}}} := (1_W \otimes 1_S \otimes \epsilon_W) \circ (\overline{\semantics{v}} \otimes \overline{\semantics{np}})$. Diagrammatically, we have:

\begin{center}
\tikzfig{V-NP}
\end{center}

\item $\overline{\semantics{\mbox{NP VP}}} := (\epsilon_W \otimes 1_S) \circ (\overline{\semantics{np}} \otimes \overline{\semantics{vp}})$. Diagrammatically, we have:

\begin{center}
\tikzfig{S}
\end{center}
\end{itemize}


%\noindent
%The  diagrammatic semantics of  a determiner `Det'  and its symbolic representation, as a compact closed morphism,  are as follows:
%
%\begin{center}
%\tikzfig{Q-Sbj-Frob} \qquad $(1_W \otimes \delta_W) \circ (1_W \otimes Det \otimes \mu_W) \circ (1_W \otimes \eta_W \otimes 1_W) \circ \eta_W $
%\end{center}
%




In the  abstract setting the meaning and semantic interpretation of sentences are the same: they both are represented by the object $S$.  In the next section we show how to instantiate this setting to a concrete relational setting where meaning can be defined to be true or false.  Here, we provide semantics  interpretations for sentences with a quantified phrase at their subject and object position.  

The interpretation of a  sentence with a  quantified phrase in  subject position  and its simplified forms are as follows:


\begin{minipage}{20cm}
\begin{minipage}{7cm}
\tikzfig{Q-Sbj-Frob-Sent}
\end{minipage}
\ $\cong$ \ \qquad
\begin{minipage}{5cm}
 \tikzfig{Q-Sbj-Norm}
\end{minipage}
\end{minipage}


\noindent
The symbolic representation of the simplified   diagram above is as follows:
\[
(\epsilon_W \otimes 1_S) \circ (\overline{\semantics{d}} \otimes  \mu_W \otimes 1_S) \circ (\delta_W \otimes 1_{W \otimes S} \otimes \epsilon_W)  \circ (\overline{\semantics{n}} \otimes \overline{\semantics{v}} \otimes \overline{\semantics{np}})
\]
Intuitively,   the determiner first makes a copy of the subject (via the Frobenius $\delta$ map), so now we have two copies of the subject. One of these is being unified with the subject argument of the verb (via the Frobenius $\mu$ map). In set-theoretic terms this is the intersection of the interpretations of subject and subjects-of-verb. The other copy is being inputted to the determiner map $\overline{\semantics{d}}$ and will produce a modified noun based on the meaning of the determiner.  The last step is the application of the unification to the output of $\overline{\semantics{d}}$. Set theoretically, this step will decide whether the intersection of the subject-of-verb and the noun belongs to the interpretation of the quantified noun. 


A sentence with a  quantified phrase in  object position is generated by the rule   `NP V Det N'. Its diagrammatic meaning   and its simplified form are as follows:



\begin{minipage}{20cm}
\begin{minipage}{7cm}
\tikzfig{Q-Obj-Frob-Sent}
\end{minipage}
\ $\cong$ \ \qquad
\begin{minipage}{5cm}
 \tikzfig{Q-Obj-Norm}
\end{minipage}
\end{minipage}


\noindent
The symbolic representation of the simplified  diagram above is as follows:
\[
(1_S \otimes \epsilon_W) \circ (1_S \otimes \mu_W \otimes \overline{\semantics{d}}) \circ (\epsilon_W \otimes 1_{S \otimes W} \otimes \delta_W) \circ (\overline{\semantics{np}} \otimes \overline{\semantics{v}} \otimes \overline{\semantics{n}})
\]
Intuitively,   the determiner first makes a copy of the object (via the Frobenius $\delta$ map), so now we have two copies of the object. One of these is being unified with the object argument of the verb (via the Frobenius $\mu$ map). In set-theoretic terms this is the intersection of the interpretations of object and objects-of-verb. The other copy is being inputted to the determiner map $Det$ and will produce a modified noun based on the meaning of the determiner.  The last step is the application of the unification to the output of $Det$. Set theoretically, this step will decide whether the intersection of the object of the verb and the noun belongs to the interpretation of the quantified noun. 


Putting the two cases together, the semantic interpretation  of a sentence with  quantified phrases both at subject and at an object position  has the following simplified form:


\begin{center}
\begin{minipage}{7cm}
\tikzfig{Q-Frob-Sbj-Obj-Norm}
\end{minipage}
\end{center}

\noindent
The symbolic representation of the above diagram is as follows:
\[{\small
(\epsilon_W \otimes 1_S \otimes \epsilon_W) \circ (\overline{\semantics{d}} \otimes  \otimes \mu_W \otimes 1_S \otimes \mu_W \otimes \overline{\semantics{d}}) \circ (\delta_W \otimes 1_{W \otimes S \otimes W} \otimes \delta_W)} \circ (\overline{\semantics{n}} \otimes \overline{\semantics{v}} \otimes \overline{\semantics{n}})
\]


