\section{Diagrammatic  Compact Closed Semantics}

%%% Local Variables: 
%%% mode: latex
%%% TeX-master: "Quant"
%%% End: 

Following the terminology and notation of \cite{BarwiseCooper81}, given a phrase containing a quantifier followed by a noun, that is `Q noun', we call `Q' a determiner and the phrase `Q noun' a quantified  phrase.Hence,  a quantified phrase is a noun phrase which is created by the application of a determiner to a noun phrase.   We suggest the following  diagrammatic semantics for  a determiner $Det$:

\begin{center}
\tikzfig{Q-Sbj-Frob}
\end{center}

\noindent
It corresponds to the following compact closed categorical morphism:
{
\[
(1_N \otimes \delta_N) \circ (1_N \otimes Det \otimes \mu_N) \circ (1_N \otimes \eta_N \otimes 1_N) \circ \eta_N 
\]}

\noindent
The meaning of the sentence with a quantified phrase in a subject position and its normalised form are as follows:


\begin{minipage}{20cm}
\begin{minipage}{7cm}
\tikzfig{Q-Sbj-Frob-Sent}
\end{minipage}
\ $\leadsto$ \
\begin{minipage}{5cm}
 \tikzfig{Q-Sbj-Norm}
\end{minipage}
\end{minipage}


\noindent
The symbolic representation of the normal form diagram is as follows:
\[{\small
(\epsilon_N \otimes 1_S) \circ (Det \otimes  \mu_N \otimes 1_S) \circ (\delta_N \otimes 1_{N \otimes S})} (\ov{\text{Sbj}} \otimes \ov{\text{Verb}} \otimes \ov{\text{Obj}})
\]


The intuitive justification  is that the determiner first makes a copy of the subject (via the Frobenius $\delta$ map), so now we have two copies of the subject. One of these is being unified with the subject argument of the verb (via the Frobenius $\mu$ map). In set-theoretic terms this is the intersection of the interpretations of subject and subjects-of-verb. The other copy is being inputted to the determiner map $Det$ and will produce a modified noun based on the meaning of the determiner.  The last step is the application of the unification to the output of $Det$. Set theoretically, this step will decide whether the intersection of the subject-of-verb and the noun belongs to the interpretation of the quantified noun. 

 
The diagrams,  morphisms, and intuitions for a quantified phrase in an object position are identical.

